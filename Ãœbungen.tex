\documentclass[a4paper]{scrartcl}
\usepackage[utf8]{inputenc}
\usepackage[T1]{fontenc}
\usepackage{lmodern}
\usepackage[sc]{mathpazo}
\linespread{1.05}
\usepackage{exscale} % To scale mathematical symbols correctly while using T1
\usepackage{amsmath,amsthm,amssymb,dsfont}
\usepackage{tikz}
	\usetikzlibrary{matrix}
\usepackage[ngerman]{babel}
\usepackage{enumitem}
\usepackage{microtype} % Microtypography!
\usepackage{epigraph}
\usepackage{hyperref}

\newcommand{\D}{\mathrm{d}}
\newcommand{\DD}{\mathrm{D}}
\newcommand{\e}{\mathrm{e}}
\newcommand{\diff}{:\Longleftrightarrow}
\DeclareMathOperator{\id}{id}
\DeclareMathOperator{\Diff}{Diff}
\DeclareMathOperator{\GL}{GL}
\DeclareMathOperator{\End}{End}
\DeclareMathOperator{\pr}{pr}
\DeclareMathOperator{\Ad}{Ad}
\DeclareMathOperator{\ad}{ad}
\DeclareMathOperator{\tr}{tr}
\DeclareMathOperator{\Exp}{Exp}


\newcommand{\R}{\mathbb{R}}
\newcommand{\sC}{\mathcal{C}^{\infty}}
\newcommand{\sm}{\mathcal{F}}
\newcommand{\vf}{\mathfrak{X}}
\newcommand{\tril}{\vartriangleleft}
\newcommand{\trir}{\vartriangleright}

\title{Übungsaufgaben}
\begin{document}
	\maketitle
\setcounter{section}{-1}
	\section{Lineare Algebra und mehrdimensionale Analysis}
	\begin{enumerate}
		\item Zeige: Die Abbildung $\Theta\colon V\rightarrow V^{**}$ ist linear und injektiv.
	\end{enumerate}
	\section{Mannigfaltigkeiten}
	\begin{enumerate}
		\item Zeige: In einem Hausdorffraum sind kompakte Mengen abgeschlossen.
		\item Zeige: Die Verkettung stetiger Abbildungen ist stetig.
		\item Zeige: $f\colon X\rightarrow Y$ stetig , $K\subset X$ kompakt $\Rightarrow f(K)\subset Y$ kompakt.
		\item Zeige: Die Fortsetzung eines nichtleeren Atlanten zu einem maximalen Atlanten ist eindeutig.
		\item Zeige: Eine Karte ist eine glatte Abbildung zwischen Mannigfaltigkeiten.
		\item Konstruiere aus den Atlanten zweier Mannigfaltigkeiten $M_1$ und $M_2$ einen Atlanten der Produktmannigfaltigkeit.
		\item Zeige: Für eine Produktmannigfaltigkeit $M_1\times M_2$ sind die kanonischen Projektionen $\pi_i\colon M_1\times M_2\rightarrow M_i$ ($i=1,2$) differenzierbar und für eine weitere Mannigfaltigkeit $N$ mit Abbildung $f\colon N\rightarrow M_1\times M_2$ gilt:
		\begin{equation}
			\text{$f$ ist differenzierbar} \colon\Leftrightarrow \pi_i\circ f \text{ ist differenzierbar für } i=1,2
		\end{equation}
	\end{enumerate}
	\section{Der Tangentialraum und das Differential}
	\begin{enumerate}
		\item Zeige: $\sC_p(M)$ ist mit über die Repräsentanten definierter Addition, Skalarmultiplikation und Multiplikation eine $\mathbb R$-Algebra.
		\item Zeige: $T_pM$ ist ein Untervektorraum von $(\sC_p(M))^*$.
		\item Zeige: Das Differential ist eine wohldefinierte, lineare Abbildung.
		\item Ist $\gamma\colon I \to M$ eine glatte Kurve und $f\colon M \to N$ glatt, so ist $\tilde\gamma := f\circ\gamma \colon I \to N$ eine Kurve in $M$. Zeige: $\dot{\tilde\gamma}(s) = \left.\DD f\right|_{\gamma(s)} (\dot\gamma(s))$.
		\item Zeige: $\left.\DD(\id_M)\right|_p = \id_{T_pM}$.
		\item Zeige: Ist $f\colon M \to N$ ein Diffeomorphismus, so ist $\left.\DD f\right|_p$ für jedes $p\in M$ ein Isomorphismus, und es gilt $\left.\DD f^{-1}\right|_{f(p)} = \left(\left.\DD f\right|_p\right)^{-1}$.
		\item Zeige: Bezüglich der kanonischen Identifikation $V\equiv T_pV$ gilt $\gamma'(s) \equiv \dot\gamma(s)$ für jede Kurve $\gamma\colon I\rightarrow V$. Dabei bezeichnet $\gamma'$ die Ableitung als Differenzquotienten und $\dot{\gamma}$ den abstrakten Tangentialvektor. (Hinweis: Verwende die mehrdimensionale Kettenregel für vektorwertige Abbildungen!)
		\item Sei $f\colon M \to N$ eine glatte Abbildung zwischen differenzierbaren Mannigfaltigkeiten. Sei $p\in M$, $x$ eine Karte von $M$ um $p$ und $y$ eine Karte von $N$ um $f(p)$. Zeige: \[\mathrm{J}(y\circ f\circ x^{-1})|_{x(p)} = \left[\left(\partial_j(y\circ f\circ x^{-1})^i\right) (x(p)) \right]_{ij}.\]
		ist die Darstellungsmatrix des Differentials der Kartendarstellung $y\circ f\circ x^{-1}$ von $f$ bezüglich der Koordinatenbasen.
		\item Rechne die Transformationsformel für Tangentialvektoren nach.
		\item Für einen Vektorraum $V$ lässt sich die Identifikation $V\equiv T_pV$ folgendermaßen schreiben: 
		\[V \ni v \equiv \left.\frac{\D}{\D t}(p + tv)\right|_{t=0} \in T_pV.\]
		\item Zeige: Die Koordinatenvektoren sind durch die Ableitungen der Koordinatenlinien gegeben.
		\item Zeige: Jeder Tangentialvektor lässt sich als Ableitung einer Kurve schreiben.
	\end{enumerate}
	\section{Vektorbündel}
	\begin{enumerate}
		\item Zeige: Ein Vektorbündel ist genau dann trivial, wenn es einen globalen Rahmen besitzt.
		\item Gib das Tangentialbündel $TS^1$ der $1$-Sphäre an, indem Du $S^1$ als Teilmenge der komplexen Zahlen auffasst und durch Kurven die Tangentialräume punktweise erzeugst.
		\item Mache Dir klar, dass ein zurückgezogener Schnitt tatsächlich ein Schnitt des zurückgezogenen Bündels ist (dabei gilt es zwei Eigenschaften zu prüfen).
		\item Zeige, dass die zurückgezogenen Schnitte die Schnitte des zurückgezogenen Bündels erzeugen. (Mit bump functions!)
		\item Konstruiere aus Karten einer Mannigfaltigkeit einen Bündelatlas des Tangentialbündels.
	\end{enumerate}
	\section{Lie-Gruppen}
	\begin{enumerate}
		\item Mache Dir noch einmal die Schritte zur Konstruktion der Lie-Algebra bewusst und entwickle ein konzeptionelles Verständnis.
		\item Zeige, dass der Kommutator von linksinvarianten Vektorfeldern linksinvariant ist.
		\item Berechne eine Basis der linksinvarianten Vektorfelder von $S^1$. Hinweis: Betrachte dazu $S^1$ wie zuvor als Teilmenge von $\mathbb{C}$ und überlege Dir zuerst anschaulich, wie das Differential der Linkstranslationen auf das Tangentialbündel wirkt.
		\item Beweise die Eigenschaften der Exponentialabbildung.
	\end{enumerate}
\end{document}