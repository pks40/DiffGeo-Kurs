\documentclass[a4paper]{scrartcl}
\usepackage[utf8]{inputenc}
\usepackage[T1]{fontenc}
\usepackage{lmodern}
\usepackage[sc]{mathpazo}
\linespread{1.05}
\usepackage{exscale} % To scale mathematical symbols correctly while using T1
\usepackage{amsmath,amsthm,amssymb,dsfont}
\usepackage{tikz}
	\usetikzlibrary{matrix}
\usepackage[ngerman]{babel}
\usepackage{enumitem}
\usepackage{microtype} % Microtypography!
\usepackage{epigraph}
\usepackage{hyperref}

\newcommand{\D}{\mathrm{d}}
\newcommand{\DD}{\mathrm{D}}
\newcommand{\e}{\mathrm{e}}
\newcommand{\diff}{:\Longleftrightarrow}
\DeclareMathOperator{\id}{id}
\DeclareMathOperator{\Diff}{Diff}
\DeclareMathOperator{\GL}{GL}
\DeclareMathOperator{\End}{End}
\DeclareMathOperator{\pr}{pr}
\DeclareMathOperator{\Ad}{Ad}
\DeclareMathOperator{\ad}{ad}
\DeclareMathOperator{\tr}{tr}
\DeclareMathOperator{\Exp}{Exp}


\newcommand{\R}{\mathbb{R}}
\newcommand{\sC}{\mathcal{C}^{\infty}}
\newcommand{\sm}{\mathcal{F}}
\newcommand{\vf}{\mathfrak{X}}
\newcommand{\tril}{\vartriangleleft}
\newcommand{\trir}{\vartriangleright}

\title{Übungsaufgaben}
\begin{document}
	\maketitle
	\section{Mannigfaltigkeiten}
	\begin{enumerate}
		\item Beweise alle Aussagen in Abschnitt 1.1.
		\item Zeige: Die Fortsetzung eines nichtleeren Atlanten zu einem maximalen Atlanten ist eindeutig.
		\item Zeige: Eine Karte ist eine glatte Abbildung zwischen Mannigfaltigkeiten.
		\item Zeige: Für eine Produktmannigfaltigkeit $M_1\times M_2$ sind die kanonischen Projektionen $\pi_i\colon M_1\times M_2\rightarrow M_i$ ($i=1,2$) differenzierbar und für eine weitere Mannigfaltigkeit $N$ mit Abbildung $f\colon N\rightarrow M_1\times M_2$ gilt:
		\begin{equation}
			\text{$f$ ist differenzierbar} \colon\Leftrightarrow \pi_i\circ f \text{ ist differenzierbar für } i=1,2
		\end{equation}
	\end{enumerate}
	\section{Vektorbündel}
	\begin{enumerate}
		\item Zeige: Ein Vektorbündel ist genau dann trivial, wenn es einen globalen Rahmen besitzt.
		\item Gib das Tangentialbündel $TS^1$ der $1$-Sphäre an, indem Du $S^1$ als Teilmenge der komplexen Zahlen auffasst und durch Kurven die Tangentialräume punktweise erzeugst.
		\item Mache Dir klar, dass ein zurückgezogener Schnitt tatsächlich ein Schnitt des zurückgezogenen Bündels ist (dabei gilt es zwei Eigenschaften zu prüfen).
		\item Zeige, dass die zurückgezogenen Schnitte die Schnitte des zurückgezogenen Bündels erzeugen. (Mit bump functions!)
		\item Konstruiere aus Karten einer Mannigfaltigkeit einen Bündelatlas des Tangentialbündels.
	\end{enumerate}
	\section{Lie-Gruppen}
	\begin{enumerate}
		\item Mache Dir noch einmal die Schritte zur Konstruktion der Lie-Algebra bewusst und entwickle ein konzeptionelles Verständnis.
		\item Zeige, dass der Kommutator von linksinvarianten Vektorfeldern linksinvariant ist.
		\item Berechne eine Basis der linksinvarianten Vektorfelder von $S^1$. Hinweis: Betrachte dazu $S^1$ wie zuvor als Teilmenge von $\mathbb{C}$ und überlege Dir zuerst anschaulich, wie das Differential der Linkstranslationen auf das Tangentialbündel wirkt.
		\item Beweise die Eigenschaften der Exponentialabbildung.
	\end{enumerate}
\end{document}