\documentclass[a4paper,headsepline,headheight=30pt,numbers=enddot]{scrartcl}
\usepackage[utf8]{inputenc}
\usepackage[T1]{fontenc}
\usepackage{lmodern}
\usepackage[sc]{mathpazo}
\linespread{1.05}
\usepackage{exscale} % To scale mathematical symbols correctly while using T1
\usepackage{amsmath,amsthm,amssymb,dsfont}
\usepackage[ngerman]{babel}
\usepackage{enumitem}
\usepackage{microtype} % Microtypography!
\usepackage{hyperref}

\usepackage{scrlayer-scrpage}
\ihead{\normalfont CdE-WinterAkademie 2018/19\\Einführung in die Differentialgeometrie: Übungen}
\automark{section}
\ohead{\normalfont\rightmark}
\chead{}
\cfoot{}
\setkomafont{section}{\normalfont\bfseries\sffamily\LARGE\centering}

\newcommand{\D}{\mathrm{d}}
\newcommand{\DD}{\mathrm{D}}
\newcommand{\e}{\mathrm{e}}
\newcommand{\diff}{:\Longleftrightarrow}
\DeclareMathOperator{\id}{id}
\DeclareMathOperator{\Diff}{Diff}
\DeclareMathOperator{\GL}{GL}
\DeclareMathOperator{\End}{End}
\DeclareMathOperator{\pr}{pr}
\DeclareMathOperator{\Ad}{Ad}
\DeclareMathOperator{\ad}{ad}
\DeclareMathOperator{\tr}{tr}
\DeclareMathOperator{\Exp}{Exp}


\newcommand{\R}{\mathbb{R}}
\newcommand{\sC}{\mathcal{C}^{\infty}}
\newcommand{\sm}{\mathcal{F}}
\newcommand{\vf}{\mathfrak{X}}

\title{Übungsaufgaben zu \glqq Einführung in die Differentialgeometrie\grqq}
\author{Benjamin Haake, Philip Schwartz}
\date{}
\begin{document}

\setcounter{section}{-1}
	\section[LinA und Ana]{Lineare Algebra und mehrdimensionale Analysis}
	\begin{enumerate}
		\item Beweise Lemma 0.2.11.
	\end{enumerate}

%%%%%%%%%%%%%%%%%%%%%%%%%%%%%%%%%%%%%%%%%%%%%%%%%%%%%%%%%%%%%%%%%%%
	\newpage
	\section{Mannigfaltigkeiten}
	\subsection*{Übungen aus dem Skript}
	\begin{enumerate}
		\item Zeige: In einem Hausdorffraum sind kompakte Mengen abgeschlossen (Lemma 1.1.7).
		\item Zeige: Die Verkettung stetiger Abbildungen ist stetig (Lemma 1.1.10).
		\item Zeige: $f\colon X\rightarrow Y$ stetig , $K\subset X$ kompakt $\Rightarrow f(K)\subset Y$ kompakt (Lemma 1.1.11).
		\item Zeige: Die Fortsetzung eines nichtleeren Atlanten zu einem maximalen Atlanten ist eindeutig (Lemma 1.2.5).
		\item Zeige: Die Verkettung glatter Abbildungen von Mannigfaltigkeiten ist glatt (Lemma 1.3.2).
		\item Zeige: Eine Karte ist ein Diffeomorphismus zwischen Mannigfaltigkeiten (Lemma 1.3.4).
	\end{enumerate}
	\subsection*{Zusätzliche Übungen}
	\begin{enumerate}
		\item Konstruiere aus den Atlanten zweier Mannigfaltigkeiten $M_1$ und $M_2$ einen Atlanten der Produktmannigfaltigkeit.
		\item Zeige: Für eine Produktmannigfaltigkeit $M_1\times M_2$ sind die kanonischen Projektionen $\pi_i\colon M_1\times M_2\rightarrow M_i$ ($i=1,2$) glatt und für eine weitere Mannigfaltigkeit $N$ mit Abbildung $f\colon N\rightarrow M_1\times M_2$ gilt:
		\begin{equation}
		\text{$f$ ist glatt} \colon\Leftrightarrow \pi_i\circ f \text{ ist glatt für } i=1,2
		\end{equation}
		\item Wir betrachten die komplexen Zahlen $\mathbb{C}$ als $2$-dimensionale Mannigfaltigkeit. Neben der üblichen Identifikation $\mathbb{C}\cong\R^2$ sind die \emph{Polarkoordinaten} eine natürliche Wahl. Dazu betrachten wir die offene Menge $U = \mathbb{C}\setminus\{z\mid \mathrm{Re}\,z\leq 0, \mathrm{Im}\,z=0\}$ zur Konstruktion von Karten.  Die Karte ist dann $x\colon U\rightarrow (-\pi,\pi)\times (0,\infty), z\mapsto (\lvert z\rvert, \mathrm{Arg}(z))$ und die inverse Abbildung ist $x^{-1}\colon (-\pi,\pi)\times (0,\infty)\rightarrow U, (\varphi,r)\mapsto re^{i\phi}$. Mache Dir klar, dass $S^1$ eine Untermannigfaltigkeit von $\mathbb{C}$ ist.
	\end{enumerate}

%%%%%%%%%%%%%%%%%%%%%%%%%%%%%%%%%%%%%%%%%%%%%%%%%%%%%%%%%%%%%%%%%%%
	\newpage
	\section[Der Tangentialraum]{Der Tangentialraum und das Differential}
	\subsection*{Übungen aus dem Skript}
	\begin{enumerate}
		\item Zeige: $\sC_p(M)$ ist mit über die Repräsentanten definierter Addition, Skalarmultiplikation und Multiplikation eine $\mathbb R$-Algebra. (Prop. 2.1.2)
		\item Zeige: $T_pM$ ist ein Untervektorraum von $(\sC_p(M))^*$. (Def. 2.1.3)
		\item Zeige: Ist $\gamma\colon I\to M, \gamma(s) = p$ eine konstante Kurve, so ist der abstrakte Geschrindigkeitsvektor $\dot\gamma(s) = 0$.
		\item Zeige: Das Differential ist eine wohldefinierte, lineare Abbildung. (Prop. 2.2.2)
		\item Ist $\gamma\colon I \to M$ eine glatte Kurve und $f\colon M \to N$ glatt, so ist $\tilde\gamma := f\circ\gamma \colon I \to N$ eine Kurve in $M$. Zeige: $\dot{\tilde\gamma}(s) = \left.\DD f\right|_{\gamma(s)} (\dot\gamma(s))$. (Bem. 2.2.3)
		\item Beweise Lemma 2.2.5 über das Zusammenspiel des Differentials mit Diffeomorphismen. Tipp: Verwende die Kettenregel.
		\item Zeige Lemma 2.3.6: Bezüglich der kanonischen Identifikation $V\equiv T_pV$ gilt $\gamma'(s) \equiv \dot\gamma(s)$ für jede Kurve $\gamma\colon I\rightarrow V$. Dabei bezeichnet $\gamma'$ die Ableitung als Differentialquotienten und $\dot{\gamma}$ den abstrakten Tangentialvektor. (Hinweis: Verwende die mehrdimensionale Kettenregel für vektorwertige Abbildungen!)
		\item Sei $f\colon M \to N$ glatt, $x$ eine Karte von $M$ und $y$ eine Karte von $N$. Zeige, dass die Darstellungsmatrix des Differentials von $f$ bzgl. der Koordinatenbasen die Jacobi-Matrix der Kartendarstellung von $f$ ist. (Lemma 2.4.8)
		\item Rechne die Transformationsformel für Tangentialvektoren nach. (Lemma 2.4.10)
		\item Beweise, dass sich die Identifikation $V\equiv T_pV$ für einen Vektorraum $V$ folgendermaßen schreiben lässt (Lemma 2.5.2): 
		\[V \ni v \equiv \left.\frac{\D}{\D t}(p + tv)\right|_{t=0} \in T_pV.\]
		\item Zeige: Die Koordinatenvektoren sind durch die Ableitungen der Koordinatenlinien gegeben (Lemma 2.5.4).
		\item Zeige: Jeder Tangentialvektor lässt sich als Ableitung einer Kurve schreiben (Lemma 2.5.5).
	\end{enumerate}
	\subsection*{Zusätzliche Übungen}
	\begin{enumerate}
		\item Nach Lemma 2.4.10:
			Wir betrachten auf $\R^2$ die offene Menge $U = \R^2 \setminus\{(x,y) \mid x \le 0\}$. Die \emph{Polarkoordinaten} auf $\R^2$ sind die Karte $\Phi = (\varphi,r) \colon U \to (-\pi,\pi) \times (0,\infty)$, die durch die Umkehrabbildung $\Phi^{-1}\colon (-\pi,\pi)\times(0,\infty), (\varphi,r) \mapsto (r\cos(\varphi), r\sin(\varphi))$ gegeben ist (dabei haben wir die \glqq böse Notation\grqq\ benutzt, die Punkte im Bildbereich der Karte genauso zu nennen wie die Kartenabbildungen selbst). Berechne mithilfe der Transformationsformel die Koordinatenvektoren $\left.\frac{\partial}{\partial\varphi}\right|_p$, $\left.\frac{\partial}{\partial r}\right|_p$ in der Koordinatenbasis der Standardkoordinaten auf $\R^2$.
		\item Nach Abschnitt 2.5:
			Fasse $S^1$ als Untermannigfaltigkeit der komplexen Zahlen auf und erzeuge die Tangentialräume $T_pS^1$ durch Kurven.
		\item Nach Lemma 2.6.2:
			Sei $M \subset \R^n$ eine eingebettete Untermannigfaltigkeit, und sei $(y,U)$ eine Karte von $M$. Die durch die Identität gegebene Karte von $\R^n$ bezeichnen wir mit $x$. Berechne die Koordinatenvektoren $\left.\frac{\partial}{\partial y^i}\right|_p$ so konkret wie möglich als Vektoren in $T_pM \subset T_p\R^n \equiv \R^n$.
		\item Nach Abschnitt 2.6: Zeige mit dem Satz vom regulären Wert, dass die Sphären $S^n$ Untermannigfaltigkeiten der euklidischen Räume $\R^n$ sind.
	\end{enumerate}

%%%%%%%%%%%%%%%%%%%%%%%%%%%%%%%%%%%%%%%%%%%%%%%%%%%%%%%%%%%%%%%%%%%
	\newpage
	\section[Das Tensorprodukt]{Multilineare Algebra: Das Tensorprodukt}
	\begin{enumerate}
		\item Beweise Lemma 3.1.2 über die Eigenschaften des Tensorprodukts $V\otimes W$.
		\item Zu Satz 3.1.3: Sei $B\colon V\times W \to X$ eine bilineare Abbildung, und die Abbildung $f_B$ definiert durch 
			\[f_B\left(\sum_{i=1}^k v_i \otimes w_i\right) := \sum_{i=1}^k B(v_i,w_i).\]
			Zeige, dass $f_B\colon V\otimes W\rightarrow X$ wohldefiniert und linear ist.
		\item Zu Lemma 3.1.5: Zeige, dass $V\otimes W\rightarrow W\otimes V,\quad v\otimes w \mapsto w\otimes v$ ein Isomorphismus ist.
		\item Zu Def. 3.3.1: Beweise die Transformationsformel für Tensorkomponenten unter Basiswechsel.
		\item Beweise Lemma 3.3.3 über die Komponentendarstellungen von Operationen mit Tensoren.
	\end{enumerate}


%%%%%%%%%%%%%%%%%%%%%%%%%%%%%%%%%%%%%%%%%%%%%%%%%%%%%%%%%%%%%%%%%%%
	\stepcounter{section}
	\newpage
	\section{Vektorbündel}
	\subsection*{Übungen aus dem Skript}
	\begin{enumerate}
		\item Konstruiere aus Karten einer Mannigfaltigkeit einen Bündelatlas des Tangentialbündels (Beispiel 5.2.3). 
		\item Zeige Korollar 5.2.9: Ein Vektorbündel ist genau dann trivial, wenn es einen globalen Rahmen besitzt.
		\item Mache Dir klar, dass ein zurückgezogener Schnitt tatsächlich ein Schnitt des zurückgezogenen Bündels ist (dabei gilt es zwei Eigenschaften zu prüfen).
		\item Zeige, dass die zurückgezogenen Schnitte die Schnitte des zurückgezogenen Bündels erzeugen. (Lemma 5.3.10; Hinweis: Betrachte lokale Rahmen und verwende Hutfunktionen.)
	\end{enumerate}
	\subsection*{Weitere Übungen}
	\begin{enumerate}
		\item Beweise, dass das Möbiusbündel nicht trivialisierbar ist (Hinweis: Verwende Korollar 5.2.9 und den Zwischenwertsatz unter Verwendung von Trivialisierungen).
		\item Überlege Dir, dass die Whitney-Summe zweier Möbiusbündel trivialisierbar ist.
	\end{enumerate}
%%%%%%%%%%%%%%%%%%%%%%%%%%%%%%%%%%%%%%%%%%%%%%%%%%%%%%%%%%%%%%%%%%%
	\newpage
	\section{Vektorfelder}
	\subsection*{Übungen aus dem Skript}
	\begin{enumerate}
		\item Zeige, dass der Kommutator eine Derivation ist, also ein Vektorfeld definiert. (Def. 6.2.2)
		\item Beweise die Eigenschaften des Kommutators. (Lemma 6.2.3)
		\item Beweise Lemma 6.3.4 über Pushforwards von Ableitungen.
		\item Beweise die Formel für die Koordinatendarstellung von Pushforwards (Lemma 6.3.6).
		\item Beweise die Skalierungseigenschaft von Integralkurven (Lemma 6.4.3).
	\end{enumerate}
	\subsection*{Weitere Übungen}
	\begin{enumerate}
		\item Nach Abschnitt 6.2: Wir betrachten $\R^2$ mit der Standard-Karte $(x,y)$ und die beiden Vektorfelder $X := \frac{\partial}{\partial x}, Y := x \frac{\partial}{\partial x} + y \frac{\partial}{\partial y}$. Berechne den Kommutator $[X,Y]$.
		\item Nach Abschnitt 6.3: Seien $X,Y$ wie in der vorherigen Aufgabe. Bestimme die Flüsse von $X$ und $Y$. Berechne die Lie-Ableitungen $\mathcal L_X Y$ und $\mathcal L_Y X$ anhand der Definition über den Fluss.
	\end{enumerate}

%%%%%%%%%%%%%%%%%%%%%%%%%%%%%%%%%%%%%%%%%%%%%%%%%%%%%%%%%%%%%%%%%%%
	\newpage
	\section{Lie-Gruppen}
	\subsection*{Übungen aus dem Skript}
	\begin{enumerate}
		\item Mache Dir noch einmal die Schritte zur Konstruktion der Lie-Algebra bewusst und entwickle ein konzeptionelles Verständnis.
		\item Zeige, dass der Kommutator von linksinvarianten Vektorfeldern linksinvariant ist (Satz 7.1.4).
		\item Beweise die Eigenschaften der Exponentialabbildung (Lemma 7.2.2).
	\end{enumerate}
	\subsection*{Weitere Übungen}
	\begin{enumerate}
		\item Zu Beispiel 7.1.2: Zeige, dass $\mathrm{SL}(n) \subset \mathrm{GL}(n)$ eine Untermannigfaltigkeit ist, und zeige damit, dass $\mathrm{SL}(n)$ eine Lie-Gruppe ist. Berechne außerdem den Tangentialraum $T_\mathds{1}\mathrm{SL}(n) \subset T_\mathds{1}\mathrm{GL}(n) = \End(\R^n) = M(n\times n)$. (Tipp: Benutze den Satz vom regulären Wert!)
		\item Zu Beispiel 6.1.7: Zeige, dass für $A,B \in \mathfrak{gl}(n) = \End(\R^n) = M(n\times n)$ die Lie-Klammer durch den Kommutator von Matrizen gegeben ist, $[A,B] = AB - BA$. (Anleitung: Bestimme $L(A) \in \vf^L(\mathrm{GL}(n))$ und rechne dann die Definition der Lie-Klammer nach. Du musst in Koordinaten rechnen.)
		\item Bestimme die Exponentialabbildung der Lie-Gruppe $(\R,+)$.
		\item Bestimme die Exponentialabbildung der Lie-Gruppe $(\R\setminus\{0\},\cdot)$.
		\item Berechne eine Basis der linksinvarianten Vektorfelder von $\mathrm{U}(1)$. Hinweis: Betrachte dazu $\mathrm{U}(1)\cong S^1$ wie zuvor als Untermannigfaltigkeit von $\mathbb{C}$ und überlege Dir zuerst anschaulich, wie das Differential der Linkstranslationen auf das Tangentialbündel wirkt.
		\item Zeige: Lie-Gruppen sind parallelisierbar (d.\,h. ihr Tangentialbündel ist trivial).
	\end{enumerate}
	
%%%%%%%%%%%%%%%%%%%%%%%%%%%%%%%%%%%%%%%%%%%%%%%%%%%%%%%%%%%%%%%%%%%
	\newpage
	\section{Tensorbündel und Tensorfelder}
	\subsection*{Übungen aus dem Skript}
	\begin{enumerate}
		\item Beweise Satz 8.2.6: Die durch den Pullback definierte Lie-Ableitung von Tensoren ist eine Tensorderivation.
	\end{enumerate}
\end{document}