% !TeX spellcheck = de_DE
\documentclass[a4paper]{scrreprt}
\usepackage[utf8]{inputenc}
\usepackage[T1]{fontenc}
\usepackage{lmodern}
\usepackage[sc]{mathpazo}
\linespread{1.05}
\usepackage{exscale} % To scale mathematical symbols correctly while using T1
\usepackage{amsmath,amsthm,amssymb,dsfont}
\usepackage{tikz}
	\usetikzlibrary{matrix}
\usepackage[ngerman]{babel}
\usepackage{enumitem}
\usepackage{microtype} % Microtypography!
\usepackage{epigraph}
\usepackage{hyperref}
\usepackage{todonotes}

\setlength{\epigraphwidth}{0.6\textwidth}

\numberwithin{equation}{chapter}

\newcommand{\D}{\mathrm{d}}
\newcommand{\DD}{\mathrm{D}}
\newcommand{\e}{\mathrm{e}}
\newcommand{\diff}{:\Longleftrightarrow}
\DeclareMathOperator{\id}{id}
\DeclareMathOperator{\Diff}{Diff}
\DeclareMathOperator{\GL}{GL}
\DeclareMathOperator{\pr}{pr}

\newcommand{\R}{\mathbb{R}}
\newcommand{\sC}{\mathcal{C}^{\infty}}
\newcommand{\sm}{\mathcal{F}(M)}
\newcommand{\vf}{\mathfrak{X}(M)}
\newcommand{\tril}{\vartriangleleft}
\newcommand{\trir}{\vartriangleright}

\theoremstyle{definition}
\newtheorem{defn}{Definition}[section]
\newtheorem{lemma}[defn]{Lemma}
\newtheorem{prop}[defn]{Proposition}
\newtheorem{satz}[defn]{Satz}
\newtheorem{kor}[defn]{Korollar}
\newtheorem{bem}[defn]{Bemerkung}
\newtheorem{bsp}[defn]{Beispiel}
\newtheorem{nota}[defn]{Notation}

% Kommentare
\newcommand{\kommP}[2][noinline]{\todo[#1,color=green!40]{#2}}
\newcommand{\kommB}[2][noinline]{\todo[#1,color=blue!20]{#2}}

\title{Einführung in die Differentialgeometrie}
\subtitle{Kurs auf der CdE-WinterAkademie 2018/19}
\author{Benjamin Haake, Philip Schwartz}
\date{November 2018}

\begin{document}

\maketitle

%%%%%%%%%%%%%%%%%%%%%%%%%%%%%%%%%%%%%%%%%%%%%%%%%%%%%%%%%%%
\setcounter{chapter}{-1}
\chapter{Einleitung}
\epigraph{Differentialgeometrie ist die Lehre von Eigenschaften, die invariant unter Notationswechsel sind.}{\textsc{Altes chinesisches Sprichwort}}

Verallgemeinerung von Kurven, Flächen und so. Extrem wichtig innerhalb der Mathematik und auch in quasi allen Anwendungsgebieten, insb. der theoretischen Physik (ART, Eichtheorien, alles!).

\section{Bekannte Konzepte, Notationen etc.}

\subsection{Lineare Algebra}

\begin{nota}
	Für einen Vektorraum $V$ über einem Körper $K$ bezeichnet \[V^* := \mathrm{Hom}(V,K) = \{f\colon V \to K : f \text{ linear}\}\] den Dualraum von $V$.
\end{nota}

\subsection{Mehrdimensionale Analysis}
\begin{nota}
	Punkte im $\mathbb R^n$ schreiben wir als $x = (x^1, \dots, x^n)$. Die Vektoren der Standardbasis von $\mathbb R^n$ schreiben wir als $\e_1, \dots, \e_n \in \mathbb R^n$, also \[\e_i = (0,\dots,\underset{i}{1},\dots,0).\]

	Sei $U \subset \mathbb R^n$ eine offene Menge und $f\colon U \to \mathbb R$ eine differenzierbare Abbildung. Die partielle Ableitung von $f$ nach der $i$-ten Komponente schreiben wir als $\partial_i f$, d.\,h. es ist
	\[(\partial_i f)(x) = \lim_{h\to 0} \frac{f(x + h \e_i) - f(x)}{h}.\]
\end{nota}


%%%%%%%%%%%%%%%%%%%%%%%%%%%%%%%%%%%%%%%%%%%%%%%%%%%%%%%%%%%
\chapter{Mannigfaltigkeiten}
	\section{Topologie}
		\begin{defn}[Topologischer Raum]
			Ein Tupel $(X,\mathcal{T})$ bestehend aus einer Menge $X$ und einer Familie von Teilmengen $\mathcal{T}$ heißt \emph{topologischer Raum}, wenn folgende Eigenschaften erfüllt sind:
			\begin{enumerate}[label=$T$\arabic*]
				\item $\emptyset, x\in \mathcal{T}$
				\item $\forall U_1, U_2\in \mathcal{T} \text{ gilt }U_1\cap U_2\in\mathcal{T}$
				\item Für jede Familie $\lbrace U_i\rbrace_{i\in I}\subset \mathcal{T}$ gilt $\bigcup_{i\in I}U_i\in\mathcal{T}$
			\end{enumerate}
			Die Familie $\mathcal{T}$ heißt \emph{Topologie} auf der Menge $X$, ihre Elemente heißen \emph{offene Mengen}. Eine Menge heißt \emph{abgeschlossen}, wenn ihr Komplement offen ist.
		\end{defn}
		\begin{defn}[Basis einer Topologie]
			Eine Familie offener Mengen $\lbrace B_i\rbrace_{i\in I}\subset \mathcal{T}$ heißt \emph{Basis} eines topologischen Raums $(X,\mathcal{T})$, wenn sich jede offene Menge als Vereinigung von Elementen der Basis schreiben lässt, das heißt:
			\begin{equation*}
				\forall U\in\mathcal{T}\exists \lbrace B_i\rbrace_{i\in J}\subset \mathcal{T}: U=\bigcup_{i\in J}B_i
			\end{equation*}
		\end{defn}
		\begin{bsp}
			Ist $(X,d)$ ein metrischer Raum, so bilden die bzgl. der Metrik offenen Mengen eine Topologie mit den $\varepsilon$-Bällen $\lbrace B_{\varepsilon}(x)\vert \; x\in X, \varepsilon >0\rbrace$ als Basis.
		\end{bsp}
		\begin{defn}[Zweitabzählbarkeit]
			Ein topologischer Raum $(X,\mathcal{T})$ heißt \emph{zweitabzählbar}, wenn für die Topologie $\mathcal{T}$ eine abzählbare Basis existiert. 
		\end{defn}
		\begin{defn}[Kompaktheit]
			Eine Teilmenge $K\subset X$ heißt \emph{kompakt}, wenn jede offene Überdeckung (das heißt jede Familie offener Mengen $\lbrace U_i\rbrace_{i\in I}$ mit $K\subset \bigcup_{i\in I}U_i$) eine endliche Teilüberdeckung hat (also endlich viele $U_1,\ldots,U_n$ existieren sodass schon $K\subset \bigcup_{i=1}^n U_i$ gilt).
		\end{defn}
		\begin{defn}[Hausdorffraum]
			Ein topologischer Raum $(X,\mathcal{T})$ heißt \emph{hausdorffsch} oder \emph{Hausdorffraum}, falls für je zwei Punkte $x,y\in X, x\neq y$ offene Umgebungen ${U_x,U_y\in\mathcal{T}}$, ${x\in U_x}, {y\in U_y}$ existieren sodass $U_x\cap U_y=\emptyset$. Zwei Punkte können also durch offene Mengen \glqq getrennt\grqq\ werden.
		\end{defn}
		\begin{lemma}
			In einem Hausdorffraum sind kompakte Mengen abgeschlossen.
		\end{lemma}
		\begin{bsp}
			Der euklidische Raum $(\R^n,d)$ ist mit der metrischen Topologie ein topologischer Raum. Er ist hausdorffsch und zweitabzählbar.
		\end{bsp}
		\begin{defn}[Stetige Abbildung]
			Eine Abbildung $f\colon (X,\mathcal{T}_X)\rightarrow (Y,\mathcal{T}_Y)$ zwischen topologischen Räumen heißt \emph{stetig}, wenn Urbilder offener Mengen offen sind:
			\begin{equation*}
				U\in\mathcal{T}_Y \Rightarrow f^{-1}(U)\in\mathcal{T}_X
			\end{equation*}
			Äquivalent dazu ist, dass Urbilder abgeschlossener Mengen abgeschlossen sind (da Komplementbildung mit dem Urbild vertauscht).
		\end{defn}
		Die Topologien werden im Folgenden in der Notation unterdrückt.
		\begin{lemma}
			Die Verkettung stetiger Abbildungen ist stetig.
		\end{lemma}
		\begin{lemma}
			Das Bild einer kompakten Menge unter einer stetigen Abbildung ist kompakt:
			\begin{equation*}
				f\colon X\rightarrow Y \text{ stetig }, K\subset X \text{ kompakt }\Rightarrow f(K)\subset Y \text{ kompakt}
			\end{equation*}
		\end{lemma}
		\begin{defn}[Homöomorphismus]
			Eine stetige Abbildung $f\colon X\rightarrow Y$ heißt \emph{Homöomorphismus}, wenn sie bijektiv und ihre Umkehrabbildung stetig ist.
		\end{defn}
	\section{Mannigfaltigkeiten}
		\begin{defn}
			Ein topologischer Raum $(M,\mathcal{T})$ heißt \emph{topologische Mannigfaltigkeit}, wenn er hausdorffsch, zweitabzählbar und lokal euklidisch ist, das heißt:
			Es existiert eine natürliche Zahl $n\in \mathbb{N}$ (Dimension), sodass für alle Punkte $x\in M$ offene Umgebungen $U\subset M, V\subset \R^n$ mit $x\in U $ und ein Homöomorphismus $ \varphi\colon U\rightarrow V$ existieren.

			Die Dimension der Mannigfaltigkeit ist wohldefiniert (Beachte, dass $n$ unabhängig von der Wahl des Punktes existieren muss).

			Das Tupel $(\varphi,U)$ heißt \emph{Karte} (wobei manchmal die Menge $U$ unterdrückt wird). Eine Familie von Karten, die die Mannigfaltigkeit überdecken, heißt \emph{Atlas} $\mathcal{A}$.
		\end{defn}
		\begin{bsp}\hfill 
			\begin{enumerate}
				\item Der euklidische Raum $\R^n$ ist eine $n$-dimensionale topologische Mannigfaltigkeit, beispielsweise mit dem Atlas $\lbrace (id_{\R^n},\R^n)\rbrace$. 
				%\item Das kartesische Produkt zweier topologischer Mannigfaltigkeiten (der Dimensionen $n$ und $m$) ist eine topologische Mannigfaltigkeit (der Dimension $n + m$).
				\item Für topologische Mannigfaltigkeiten gleicher Dimension ist die disjunkte Vereinigung eine topologische Mannigfaltigkeit (\emph{topologische Summe}).
			\end{enumerate}
		\end{bsp}
		\begin{defn}
			Für zwei Karten $\varphi_i\colon U_i\rightarrow V_i, i\in\lbrace 1,2 \rbrace$ heißt die folgende Abbildung \emph{Kartenwechsel}:
			\begin{equation*}
				\varphi_2\circ\varphi_1^{-1}\vert_{\varphi_1(U_1\cap U_2)}\colon \varphi_1(U_1\cap U_2)\rightarrow \varphi_2(U_1\cap U_2)
			\end{equation*}
			Da die Karten Homöomorphismen sind, sind auch alle Kartenwechsel Homöomorphismen.
		\end{defn} 
		\begin{defn}[Differenzierbare Mannigfaltigkeit]\hfill
			\begin{itemize}
				\item Zwei Karten heißen \emph{verträglich}, wenn ihr Kartenwechsel ein (glatter) Diffeomorphismus ist (das heißt, der Kartenwechsel und sein Inverses sind glatt).
				\item Ein Atlas heißt \emph{differenzierbar}, wenn alle seine Karten paarweise verträglich sind.
				\item Ein differenzierbarer Atlas heißt \emph{maximal}, wenn es keine Karte gibt, die mit allen Karten des Atlanten verträglich und nicht in ihm enthalten ist.
				\item Eine \emph{differenzierbare Struktur} auf einer topologischen Mannigfaltigkeit $M$ ist ein maximaler differenzierbarer Atlas $\mathcal{A}$ und das Tupel $(M,\mathcal{A})$ heißt \emph{differenzierbare Mannigfaltigkeit}.
				\kommP{Sollten erwähnen, dass, wenn Mf. gegeben, Karten immer als aus der Struktur gemeint sind.}
			\end{itemize}
		\end{defn}
		\begin{bsp}\hfill 
			\begin{enumerate}
				\item Der euklidische Raum $\R^n$ ist eine $n$-dimensionale differenzierbare Mannigfaltigkeit, der oben genannte Atlas muss dazu aber erweitert werden!
				\kommP{Erwähnen, dass jeder dfb. Atlas einen eindeutige dfb. Struktur induziert, mit der er verträglich ist? Übungsaufgabe?}
				\item Reelle, endlich dimensionale Vektorräume sind differenzierbare Mannigfaltigkeiten.
				%\item Das kartesische Produkt zweier differenzierbarer Mannigfaltigkeiten (der Dimensionen $n$ und $m$) ist eine differenzierbare Mannigfaltigkeit (der Dimension $n+m$).
				\item Für differenzierbare Mannigfaltigkeiten gleicher Dimension ist die disjunkte Vereinigung eine differenzierbare Mannigfaltigkeit (\emph{topologische Summe}).
			\end{enumerate}
		\end{bsp}
	\section{Differenzierbare Abbildungen}
		Seien von nun an $M,N$ differenzierbare Mannigfaltigkeiten.
		\begin{defn}[Differenzierbarkeit]
			Eine Abbildung $f\colon M\rightarrow N$ zwischen differenzierbaren Mannigfaltigkeiten heißt \emph{glatt}, wenn für alle Punkte $p\in M$ Karten $(\varphi,U)$ um  $p$ und $(\psi,V)$ um $f(p)$ existieren, sodass $f(U)\subset V$ und die folgende Abbildung glatt ist:
			\kommP{Evtl. $C^k$ definieren?}
			\begin{equation}
				\psi\circ f \circ \varphi^{-1}\colon \varphi(U)\rightarrow \psi(V)
			\end{equation}
			Diese Definition ist unabhängig von der Wahl der Karte, da Kartenwechsel glatt sind. Äquivalent kann daher auch gefordert werden, dass die obige Abbildung für alle Karten glatt ist.

			Wir schreiben $\sC(M,N)$ für die Menge der glatten Abbildungen.
		\end{defn}
		\begin{nota}
			$\sm:=\sC(M,\R)$ ist die Menge der (glatten) Funktionen auf $M$.
		\end{nota}
		\begin{defn}
			Eine Abbildung $f\in\sC(M,N)$ heißt Diffeomorphismus, wenn $f$ bijektiv ist und $f^{-1}\in\sC(N,M)$ gilt.

			Die Menge der Diffeomorphismen $\Diff(M):=\sC(M,M)$ bildet eine Gruppe.
		\end{defn}

%%%%%%%%%%%%%%%%%%%%%%%%%%%%%%%%%%%%%%%%%%%%%%%%%%%%%%%%%%%
\chapter{Der Tangentialraum, das Differential, Untermannigfaltigkeiten}

\section{Der Tangentialraum}
Idee: Was sind \glqq Richtungen\grqq\ auf einer Mannigfaltigkeit?

Anschaulich: Wenn Mf. eingebettet im $\mathbb R^n$, können wir uns eine Tangentialebene vorstellen (Bild: $S^2$ in $\mathbb R^3$). Also: Mögliche Richtungen = Ableitungen von Kurven. Aber abstrakt?

Müssen intrinsisch darüber reden können. Dabei hilft die Beobachtung, dass man Richtungen dazu benutzt, zu beschreiben, wie sich Dinge bei Bewegung in diese Richtung ändern. Also: Richtung = Richtungsableitung! Wir werden deshalb \glqq Richtungen\grqq\ als Ableitungsoperatoren formalisieren.

Idee: Ableitungen erfüllen eine Produktregel und wirken \glqq lokal\grqq.

Sei im Folgenden $M$ eine differenzierbare Mannigfaltigkeit, $\dim M = n$.

\begin{defn}[Funktionskeime]
	Sei $p \in M$. Auf der Menge $X = \{f \in \sC(U) : U \subset M \text{ offene Umgebung von } p\}$ betrachten wir die Äquivalenzrelation $\sim$, die gegeben ist durch
	\[f \sim g \diff \exists \text{ offene Umgebung } V \text{ von } p \text{ mit } f = g \text{ auf } V.\]
	Die Äquivalenzklassen bzgl. dieser Relation heißen glatte \emph{Funktionskeime} in $p$. Zwei um $p$ definierte Funktionen definieren also genau dann denselben Keim, wenn sie in einer Umgebung von $p$ übereinstimmen.

	Den Raum der Funktionskeime in $p$ schreiben wir als $\sC_p(M)$. Den Keim zu einer Funktion $f$ schreiben wir als $[f]$. Wenn keine Missverständnisse entstehen können, schreiben wir stattdessen manchmal auch $f$, um uns Notationsaufwand zu ersparen.
\end{defn}

Zu einem Funktionskeim $[f] \in \sC_p(M)$ ist der Funktionswert $[f](p) := f(p)$ wohldefiniert (warum?).

\begin{prop}
	\kommP{Das hab ich mal \glqq Proposition\grqq\ und nicht \glqq Lemma\grqq\ genannt, weil es nur im aktuellen Kontext relevant ist und man es außerhalb nicht braucht.}
	$\sC_p(M)$ ist mit über die Repräsentanten definierter Addition, Skalarmultiplikation und Multiplikation eine $\mathbb R$-Algebra.

	\begin{proof}
		Übung.
	\end{proof}
\end{prop}

\begin{defn}
	Eine \emph{Derivation} von $\sC_p(M)$ ist eine lineare Abbildung $v\colon \sC_p(M) \to \mathbb R$, die die \glqq Produktregel\grqq\ \[v(fg) = v(f) g(p) + f(p) v(g)\] erfüllt. Den Raum der Derivationen von $\sC_p(M)$ nennen wir den \emph{Tangentialraum} $T_pM$ an $M$ im Punkt $p$; die Elemente von $T_pM$ heißen auch \emph{Tangentialvektoren} in $p$.
\end{defn}

$T_pM$ ist tatsächlich ein Untervektorraum von $(\sC_p(M))^*$ (Übung).

\begin{lemma}
	Für alle $v \in T_pM$ und $c \in \mathbb R$ ist $v(c) = 0$ (dabei fassen wir $c$ als konstante Funktion bzw. den davon induzierten Funktionskeim auf).
	
	\begin{proof}
		Es ist $v(1) = v(1\cdot 1) = v(1) 1(p) + 1(p) v(1) = 2 v(1)$, also $v(1) = 0$. Da $v$ linear ist, folgt die Behauptung.
	\end{proof}
\end{lemma}

\begin{defn}
	Sei $(\varphi,U)$ eine Karte um $p$. Für $i = 1,\dots, n$ definieren wir den Tangentialvektor $\left.\frac{\partial}{\partial x^i}\right|_p \in T_pM$ durch
	\[\left.\frac{\partial}{\partial x^i}\right|_p(f) := \left(\partial_i \left(f\circ \varphi^{-1}\right)\right) (\varphi(p)).\]
\end{defn}
$\left.\frac{\partial}{\partial x^i}\right|_p$ stellt also einen Funktionskeim in der Karte $\varphi$ dar und bildet die $i$-te partielle Ableitung dieser Kartendarstellung.
\begin{prop}
	$\left.\frac{\partial}{\partial x^i}\right|_p$ ist tatsächlich eine Derivation von $\sC_p(M)$.

	\begin{proof}
		$\left.\frac{\partial}{\partial x^i}\right|_p$ ist wohldefiniert, denn wenn $[f] = [g]$ gilt, dann stimmen $f$ und $g$ in einer Umgebung von $p$ überein, und damit stimmen $f\circ \varphi^{-1}$ und $g\circ \varphi^{-1}$ in einer Umgebung von $\varphi(p)$ überein und haben insbesondere dieselben Ableitungen an der Stelle $\varphi(p)$.

		Die Linearität von $\left.\frac{\partial}{\partial x^i}\right|_p$ folgt direkt aus der Linearität von partiellen Ableitungen. Dass $\left.\frac{\partial}{\partial x^i}\right|_p$ eine Derivation ist, folgt aus der Produktregel für partielle Ableitungen:
		\begin{align*}
			\left.\frac{\partial}{\partial x^i}\right|_p (fg) &= \left(\partial_i \left((fg)\circ \varphi^{-1}\right)\right) (\varphi(p))\\
			&= \left(\partial_i \left( \left(f\circ \varphi^{-1}\right) \cdot \left(g\circ \varphi^{-1}\right) \right) \right) (\varphi(p))\\
			&= \left(\partial_i \left(f\circ \varphi^{-1}\right) \right)(\varphi(p)) \cdot \left(g\circ \varphi^{-1}\right) (\varphi(p)) \\&\quad+ \left(f\circ \varphi^{-1}\right) (\varphi(p)) \cdot \left(\partial_i \left(g\circ \varphi^{-1}\right) \right)(\varphi(p))\\
			&= \left.\frac{\partial}{\partial x^i}\right|_p (f) \cdot g(p) + f(p) \cdot \left.\frac{\partial}{\partial x^i}\right|_p (g) \qedhere
		\end{align*}
	\end{proof}
\end{prop}

Wir wollen jetzt zeigen, dass die $\left.\frac{\partial}{\partial x^i}\right|_p, i \in \{1,\dots,n\}$ eine Basis von $T_pM$ bilden. Dafür brauchen wir einen kleinen Trick:
\begin{lemma}[Hadamard-Lemma]
	Sei $U \subset \mathbb R^n$ eine sternförmige offene Menge, $x_0 \in U$ und $f \in \sC(U)$. Dann gibt es glatte Funktionen $f_i \in \sC(U), i \in \{1,\dots,n\}$, sodass $f(x) = f(x_0) + \sum_{i=1}^n f_i(x) (x^i - x_0^i)$.

	\begin{proof}
		Bla
	\end{proof}
\end{lemma}


%%%%%%%%%%%%%%%%%%%%%%%%%%%%%%%%%%%%%%%%%%%%%%%%%%%%%%%%%%%
\chapter{Vektorbündel}
	\section{Faserbündel}
		\begin{defn}[Faserbündel]
			Ein \emph{Faserbündel} $(P,M,\pi ,F)$ ist ein Tupel aus drei Mannigfaltigkeiten (dem \emph{Totalraum} $P$, der \emph{Basis} $M$ und der \emph{typischen Faser} $F$) zusammen mit einer surjektiven \emph{Projektion} $\pi\colon P\rightarrow M$, die die folgenden Eigenschaften erfüllt:

				Für alle Punkte $x\in M$ existieren eine offene Umgebung $x\in U\subset M$ und ein Diffeomorphismus $\phi\colon \pi^{-1}(U)\rightarrow U\times F$, sodass das folgende Diagramm kommutiert, also $\pi\vert_{\pi^{-1}(U)}=\pr_1\circ\phi$ (wobei $\pr_1$ die Projektion auf den ersten Faktor bezeichnet):
			\begin{center}
			\begin{tikzpicture}
				\matrix(m)[matrix of math nodes,row sep=3em,column sep=4em,minimum width=2em]{\pi^{-1}(U) & U\times F \\
     U& \\};
  				\path[-stealth]
  					(m-1-1)	edge node [above]{$\phi$} (m-1-2)
  							edge node [left]{$\pi$} (m-2-1)
  					(m-1-2) edge node [below]{$\quad\pr_1$} (m-2-1)
  					;
			\end{tikzpicture}
			\end{center}
			Eine solche Abbildung $\phi$ heißt \emph{lokale Trivialisierung}. Aufgrund der Existenz lokaler Trivialisierungen ist das Urbild jedes Punktes $x\in M$ unter der Projektion (\emph{Faser} von $x$ genannt) $P_x:=\pi^{-1}(\lbrace x\rbrace )$ diffeomorph zur typischen Faser: $P_x\cong F \; \forall x\in M$.

			Eine Familie lokaler Trivialisierungen, die $P$ überdecken, heißt \emph{Bündelatlas}.
		\end{defn}
		\begin{nota}
			Man sagt auch \glqq Faserbündel über $M$\grqq und schreibt dafür einfach $(P,\pi)$.
		\end{nota}
		Sei von nun an $(P,M,\pi ,F)$ ein Faserbündel.
		\begin{bsp}\hfill
			\begin{enumerate}
				\item Das triviale Bündel mit typischer Faser $F$ und Basis $M$ ist das kartesische Produkt $M\times F$ (genauer gesagt das Bündel $(M\times F,M,\pr_1,F))$.
				\item Für eine offene Menge $U\subset M$ und $P\vert_U:=\pi^{-1}(U)$ ist $(P\vert_U,U,\pi\vert_{P_U},F)$ ein Faserbündel.
			\end{enumerate}
		\end{bsp}
		\begin{defn}[Schnitt]
			Eine Abbildung $\sigma\colon M\rightarrow P$ heißt (globaler) \emph{Schnitt}, wenn $\pi\circ\sigma=\id_M$, also $\sigma(x)\in P_x\, \forall x\in M$. Analog heißt für eine offene Menge $U\subset M$ eine Abbildung $\sigma\colon U\rightarrow P\vert_U$ \emph{lokaler Schnitt}, wenn $\pi\circ\sigma=\id_U$.

			Die Menge aller globalen Schnitte wird mit $\Gamma(P)$ bezeichnet, die der lokalen Schnitte mit $\Gamma(P\vert_U)$.
		\end{defn}
	\section{Vektorbündel}
		\begin{defn}[Vektorbündel]
				Ein Faserbündel $(E,M,\pi,\R^n)$ heißt \emph{Vektorbündel} vom \emph{Rang} $n$, wenn $E_x$ für alle Punkte $x\in M$ ein $n$-dimensionaler Vektorraum ist. Weiter soll es einen Bündelatlas aus lokalen Trivialisierungen ${\phi\colon \pi^{-1}(U)\rightarrow U\times \R^n}$ geben, sodass die folgende Abbildung für jeden Punkt $x\in U$ ein Vektorraumisomorphismus ist:
				\begin{equation*}
					E_x\rightarrow \R^n,\; p\mapsto \pr_2(\phi(p))
				\end{equation*}
			Daraus folgt, dass die Abbildung ${\R^n\rightarrow E_x,\; v\mapsto \phi^{-1}(x,v)}$ ebenfalls ein Vektorraumisomorphismus ist.
		\end{defn}
		Analog zu Kartenwechseln lassen sich auch \emph{Trivialisierungswechsel} definieren:
		\begin{defn}
			Sei $(E,M,\pi,\R^n)$ ein Vektorbündel, $U,V\subset M$ mit $U\cap V\neq\emptyset$ und Trivialisierungen ${\phi_U\colon \pi^{-1}(U)\rightarrow U\times \R^n}, {\phi_V\colon \pi^{-1}(U)\rightarrow U\times \R^n}$.

			Die folgende Abbildung heißt \emph{Trivialisierungswechsel}:
			\begin{equation*}
				\phi_V\circ\phi_U^{-1}\colon (U\cap V)\times \R^n\rightarrow (U\cap V)\times \R^n, (q,v)\mapsto (q,\tau(q)v)
			\end{equation*}
			wobei die Einschränkung unterdrückt wurde. Die Abbildung $\tau\colon U\cap V\rightarrow \GL(n,\R)$ ist glatt und wird als \emph{Übergangsfunktion} bezeichnet.
		\end{defn}
		\begin{defn}[Vektorbündelhomomorphismen]\hfill\\
			Seien $(E_a,M_a,\pi_a,\R^{n_a}),(E_b,M_b,\pi_b,\R^{n_b})$ Vektorbündel.
			
			Ein Tupel $(F,f)$ mit $F\colon E_a\rightarrow E_b$ und $f\colon M_a\rightarrow M_b$ heißt \emph{Vektorbündelhomomorphismus}, wenn das folgende Diagramm kommutiert:
			\begin{center}
				\begin{tikzpicture}
					\matrix(m)[matrix of math nodes,row sep=3em,column sep=4em,minimum width=2em]{E_a & E_b \\
     M_a&M_b \\};
  				\path[-stealth]
  					(m-1-1)	edge node [above]{$F$} (m-1-2)
  							edge node [left]{$\pi_a$} (m-2-1)
  					(m-1-2) edge node [right]{$\pi_b$} (m-2-2)
  					(m-2-1) edge node [above]{$f$} (m-2-2)
  					;
				\end{tikzpicture}
			\end{center}
			also $\pi_b\circ F=f\circ\pi_a$.
		\end{defn}
		\begin{bsp}\hfill
			\begin{enumerate}
				\item Das triviale Bündel $M\times \R^n$ ist ein Vektorbündel vom Rang $n$.
				\item Das Tangentialbündel $TM:=\bigcup_{x\in M}T_xM$ ist ein Vektorbündel vom Rang $n:=\dim M$. Beachte jedoch, dass im Allgemeinen $TM\ncong M\times \R^n$ (es existiert also nicht unbedingt ein Vektorbündelisomorphismus zwischen dem Tangential- und dem trivialen Bündel). Im Fall $TM\cong M\times \R^n$ heißt die Mannigfaltigkeit parallelisierbar.
			\end{enumerate}
		\end{bsp}
	\section{Konstruktionen von Vektorfeldern}
		Seien $(E_a,M,\pi_a,\R^{n_a}),(E_b,M,\pi_b,\R^{n_b})$ Vektorbündel.\\
		\begin{defn}
			Das zu $(E,M,\pi,\R^{n})$ \emph{duale Bündel} $(E^{\prime},M,\pi^{\prime},(\R^n)^{\prime})$ ist gegeben durch
			\begin{align*}
				E^{\prime}&:=\bigcup_{x\in M}(E_{x})^{\prime}\\
				\pi^{\prime}&\colon E^{\prime}\rightarrow M\\
				(\pi^{\prime})^{-1}(x)&=E^{\prime}_x=(E_{x})^{\prime}\quad \forall x\in M
			\end{align*}
			und für eine Trivialisierung $(\phi,U)$ von $E$, ist die folgende Abbildung eine Trivialisierung des dualen Bündels:
			\begin{align*}
				\phi^{\prime}\colon(\pi^{\prime})^{-1}(U)&\rightarrow U\times (\R^n)^{\prime}\\
				\lambda&\mapsto\left(x,\lambda\circ(\pr_2\circ\phi)^{-1}\right)
			\end{align*}
			wobei $\lambda\in E_x^{\prime}$. Trivialisierungswechsel sind (unter Identifikation von $\R^n$ mit Spalten- und $(\R^n)^{\prime}$ mit Zeilenvektoren) durch $\tau^{\prime}=\tau^{-1}$ gegeben, sodass $(x,\lambda)\mapsto(x,\lambda\tau^{\prime}(x))$.
		\end{defn}
		\begin{bsp}
			Mit obiger Konstruktion können wir aus dem Tangentialbündel $TM$ das \emph{Kotangentialbündel} $T^{\prime}M$ gewinnen. Wie wir später sehen werden, ergeben sich in Kombination mit den folgenden Konstruktionen viele neue Strukturen.
		\end{bsp}
		\begin{defn}[Whitney-Summe]
			Die \emph{Whitney-Summe} $(E_a\oplus E_b ,M,\pi,\R^{n_a+n_b})$ ist ein Vektorbündel gegeben durch
			\begin{align*}
				E_a\oplus E_b&:=\bigcup_{x\in M}E_{a,x}\oplus E_{b,x}\\
				\pi&\colon E_a\oplus E_b\rightarrow M\\
				(\pi)^{-1}(x)&=(E_a\oplus E_b)_x=E_{a,x}\oplus E_{b,x}\quad \forall x\in M
			\end{align*}
			und für Trivialisierungen $(\phi_a,U),(\phi_b,U)$ ist die folgende Abbildung eine Trivialisierung der Whitney-Summe:
			\begin{align*}
				\phi\colon\pi^{-1}(U)&\rightarrow U\times \R^{n_a+n_b}\\
				v_a\oplus v_b&\mapsto (\pi_a(v_a), \pr_2(\phi_a(v_a))\oplus\pr_2(\phi_b(v_b)))
			\end{align*}
			Trivialisierungswechsel sind (wieder unter Verwendung von Spaltenvektoren) durch die folgende Blockdiagonalmatrix gegeben: 
			\begin{equation}
				\tau=\left(\begin{array}{cc}\tau_a&0\\0&\tau_b\\ \end{array}\right)
			\end{equation}
		\end{defn}
		\begin{defn}[Tensorprodukt]
			Das \emph{Tensorprodukt} $(E_a\otimes E_b ,M,\pi,\R^{n_a\cdot n_b})$ von Vektorbündeln ist ein Vektorbündel gegeben durch
			\begin{align*}
				E_a\otimes E_b&:=\bigcup_{x\in M}E_{a,x}\otimes E_{b,x}\\
				\pi&\colon E_a\otimes E_b\rightarrow M\\
				(\pi)^{-1}(x)&=(E_a\otimes E_b)_x=E_{a,x}\otimes E_{b,x}\quad \forall x\in M
			\end{align*}
			und für Trivialisierungen $(\phi_a,U),(\phi_b,U)$ ist die folgende Abbildung eine Trivialisierung des Tensorprodukts:
			\begin{align*}
				\phi_a\otimes\phi_b\colon\pi^{-1}(U)&\rightarrow U\times \R^{n_a\cdot n_b}\\
				v_a\otimes v_b&\mapsto (\pi_a(v_a), \pr_2(\phi_a(v_a))\otimes\pr_2(\phi_b(v_b)))
			\end{align*}
			Die Übergangsfunktionen des Tensorproduktes sind durch das Tensorprodukt der Übergangsfunktionen gegeben: $\tau=\tau_a\otimes\tau_b$.
		\end{defn}
		\todo{Tensorprodukt von VR}
		
%%%%%%%%%%%%%%%%%%%%%%%%%%%%%%%%%%%%%%%%%%%%%%%%%%%%%%%%%%%
\chapter{Vektorfelder}


%%%%%%%%%%%%%%%%%%%%%%%%%%%%%%%%%%%%%%%%%%%%%%%%%%%%%%%%%%%
\chapter{Lie-Gruppen}


%%%%%%%%%%%%%%%%%%%%%%%%%%%%%%%%%%%%%%%%%%%%%%%%%%%%%%%%%%%
\chapter{Tensorfelder}


%%%%%%%%%%%%%%%%%%%%%%%%%%%%%%%%%%%%%%%%%%%%%%%%%%%%%%%%%%%
\chapter{Differentialformen}


%%%%%%%%%%%%%%%%%%%%%%%%%%%%%%%%%%%%%%%%%%%%%%%%%%%%%%%%%%%
\chapter{Kovariante Ableitungen}


%%%%%%%%%%%%%%%%%%%%%%%%%%%%%%%%%%%%%%%%%%%%%%%%%%%%%%%%%%%
\chapter{Riemann'sche Geometrie}


%%%%%%%%%%%%%%%%%%%%%%%%%%%%%%%%%%%%%%%%%%%%%%%%%%%%%%%%%%%
\chapter{Integration auf Mannigfaltigkeiten}


\end{document}