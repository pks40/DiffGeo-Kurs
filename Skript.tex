% !TeX spellcheck = de_DE
\documentclass[a4paper]{scrreprt}
\usepackage[utf8]{inputenc}
\usepackage[T1]{fontenc}
\usepackage{lmodern}
\usepackage[sc]{mathpazo}
\linespread{1.05}
\usepackage{exscale} % To scale mathematical symbols correctly while using T1
\usepackage{amsmath,amsthm,amssymb,dsfont}
\usepackage[ngerman]{babel}
\usepackage{enumitem}
\usepackage{microtype} % Microtypography!
\usepackage{epigraph}
\usepackage{hyperref}
\usepackage{todonotes}

\setlength{\epigraphwidth}{0.6\textwidth}

\numberwithin{equation}{chapter}

\newcommand{\D}{\mathrm{d}}
\newcommand{\DD}{\mathrm{D}}
\newcommand{\e}{\mathrm{e}}

\newcommand{\R}{\mathds{R}}
\newcommand{\sC}{\mathcal{C}^{\infty}}
\newcommand{\sm}{\mathcal{F}(M)}
\newcommand{\vf}{\mathfrak{X}(M)}
\newcommand{\tril}{\vartriangleleft}
\newcommand{\trir}{\vartriangleright}

\theoremstyle{plain}
\newtheorem{defn}{Definition}[section]
\newtheorem{lemma}[defn]{Lemma}
\newtheorem{satz}[defn]{Satz}
\newtheorem{kor}[defn]{Korollar}
\newtheorem{bem}[defn]{Bemerkung}
\newtheorem{bsp}[defn]{Beispiel}
\newtheorem{nota}[defn]{Notation}

\title{Einführung in die Differentialgeometrie}
\subtitle{Kurs auf der CdE-WinterAkademie 2018/19}
\author{Benjamin Haake, Philip Schwartz}
\date{November 2018}

\begin{document}

\maketitle

%%%%%%%%%%%%%%%%%%%%%%%%%%%%%%%%%%%%%%%%%%%%%%%%%%%%%%%%%%%
\setcounter{chapter}{-1}
\chapter{Einleitung}
\epigraph{Differentialgeometrie ist die Lehre von Eigenschaften, die invariant unter Notationswechsel sind.}{\textsc{Altes chinesisches Sprichwort}}

Blabla


%%%%%%%%%%%%%%%%%%%%%%%%%%%%%%%%%%%%%%%%%%%%%%%%%%%%%%%%%%%
\chapter{Mannigfaltigkeiten}
	\section{Topologie}
		\begin{defn}[Topologischer Raum]
			Eine Tupel $(X,\mathcal{T})$ bestehend aus einer Menge $X$ und einer Familie von Teilmengen $\mathcal{T}$ heißt topologischer Raum, wenn folgende Eigenschaften erfüllt sind:
			\begin{enumerate}[label=$T$\arabic*]
				\item $\emptyset, x\in \mathcal{T}$
				\item $\forall U_1, U_2\in \mathcal{T} \text{ gilt }U_1\cap U_2\in\mathcal{T}$
				\item Für jede Familie $\lbrace U_i\rbrace_{i\in I}\subset \mathcal{T}$ gilt $\bigcup_{i\in I}U_i\in\mathcal{T}$
			\end{enumerate}
			Die Familie $\mathcal{T}$ heißt Topologie auf der Menge $X$, seine Elemente heißen offene Mengen. Eine Menge heißt abgeschlossen, wenn ihr Komplement offen ist.
		\end{defn}
		\begin{defn}[Basis einer Topologie]
			Eine Familie offener Mengen $\lbrace B_i\rbrace_{i\in I}\subset \mathcal{T}$ heißt Basis eines topologischen Raums $(X,\mathcal{T})$, wenn sich jede offene Menge als Vereinigung von Elementen der Basis schreiben lässt, das heißt:
			\begin{equation*}
				\forall U\in\mathcal{T}\exists \lbrace B_i\rbrace_{i\in J}\subset \mathcal{T}: U=\bigcup_{i\in J}B_i
			\end{equation*}
		\end{defn}
		\begin{bsp}
			Ist $(X,d)$ ein metrischer Raum, so bilden die bzgl der Metrik offenen Mengen eine Topologie mit den $\varepsilon$-Bällen $\lbrace B_{\varepsilon}(x)\vert \; x\in X, \varepsilon >0\rbrace$ als Basis.
		\end{bsp}
		\begin{defn}[Zweitabzählbarkeit]
			Ein topologischer Raum $(X,\mathcal{T})$ heißt zweitabzählbar, wenn für die Topologie $\mathcal{T}$ eine abzählbare Basis existiert. 
		\end{defn}
		\begin{defn}[Kompaktheit]
			Eine Teilmenge $K\subset X$ heißt kompakt, wenn jede offene Überdeckung (das heißt jede Familie offener Mengen $\lbrace U_i\rbrace_{i\in I}$ mit $K\subset \bigcup_{i\in I}U_i$) eine endliche Teilüberdeckung hat (also endlich viele $U_1,\ldots,U_n$ existieren sodass schon $K\subset \bigcup_{i=1}^n U_i$ gilt).
		\end{defn}
		\begin{defn}[Hausdorffraum]
			Ein topologischer Raum $(X,\mathcal{T})$ heißt hausdorffsch oder Hausdorffraum, falls für je zwei Punkte $x,y\in X, x\neq y$ offene Umgebungen $U_x,U_y\in\mathcal{T}, x\in U_x, y\in U_y$ existieren sodass $U_x\cap U_y=\emptyset$. Zwei Punkte können also durch offene Mengen \glqq getrennt \grqq werden.
		\end{defn}
		\begin{lemma}
			In einem Hausdorffraum sind kompakte Mengen abgeschlossen.
		\end{lemma}
		\begin{bsp}
			Der euklidische Raum $(\R^n,d)$ ist mit der metrischen Topologie ein topologischer Raum. Insbesondere ist er hausdorffsch und zweitabzählbar.
		\end{bsp}
		\begin{defn}[Stetige Abbildung]
			Eine Abbildung $f:(X,\mathcal{T}_X)\rightarrow (Y,\mathcal{T}_Y)$ zwischen topologischen Räumen heißt stetig, wenn Urbilder offener Mengen offen sind:
			\begin{equation*}
				U\in\mathcal{T}_Y \Rightarrow f^{-1}(U)\in\mathcal{T}_X
			\end{equation*}
			Äquivalent dazu ist, dass Urbilder abgeschlossener Mengen abgeschlossen sind (da Komplementbildung mit dem Urbild vertauscht).\\
			Die Topologien werden im Folgenden in der Notation unterdrückt.
		\end{defn}
		\begin{lemma}
			Die Verkettung stetiger Abbildungen ist stetig.
		\end{lemma}
		\begin{lemma}
			Das Bild einer kompakten Menge unter einer stetigen Abbildung ist kompakt:
			\begin{equation*}
				f:X\rightarrow Y \text{ stetig }, K\subset X \text{ kompakt }\Rightarrow f(K)\subset Y \text{ kompakt}
			\end{equation*}
		\end{lemma}
		\begin{defn}[Homöomorphismus]
			Eine stetige Abbildung $f:X\rightarrow Y$ heißt Homöomorphismus, wenn sie bijektiv und ihre Umkehrabbildung stetig ist.
		\end{defn}
	\section{Mannigfaltigkeiten}
		\begin{defn}
			Ein topologischer Raum $(M,\mathcal{T})$ heißt topologische Mannigfaltigkeit, wenn er hausdorffsch, zweitabzählbar und lokal euklidisch ist, das heißt:
			\begin{equation*}
				\exists n\in \mathds{N}: \forall x\in M\, \exists U\subset M, V\subset \R^n\text{ offen mit } x\in U \text{ und ein Homöomorphismus } \varphi:U\rightarrow V
			\end{equation*}
			Die natürliche Zahl $n$ heißt Dimension der Mannigfaltigkeit und ist wohldefiniert (beachte, dass $n$ unabhängig von der Wahl des Punktes existieren muss).\\
			Das Tupel $(\varphi,U)$ heißt Karte (wobei manchmal die Menge $U$ unterdrückt wird). Eine Familie von Karten, die die Mannigfaltigkeit überdecken, heißt Atlas $\mathcal{A}$.
		\end{defn}
		\begin{bsp}\hfill 
			\begin{enumerate}
				\item Der euklidische Raum $\R^n$ ist eine $n$-dimensionale topologische Mannigfaltigkeit, beispielsweise mit dem Atlas $\lbrace (id_{\R^n},\R^n)\rbrace$. 
				\item Das kartesische Produkt zweier topologischer Mannigfaltigkeiten (der Dimensionen $n$, $m$) ist eine topologische Mannigfaltigkeit (der Dimension $n\cdot m$).
				\item Für topologische Mannigfaltigkeiten gleicher Dimension ist die disjunkte Vereinigung eine topologische Mannigfaltigkeit.
			\end{enumerate}
		\end{bsp}
		\begin{defn}
			Für zwei Karten $\varphi_i:U_i\rightarrow V_i, i\in\lbrace 1,2 \rbrace$ heißt die folgende Abbildung Kartenwechsel:
			\begin{equation*}
				\varphi_2\circ\varphi_1^{-1}\vert_{\varphi_1(U_1\cap U_2)}:\varphi_1(U_1\cap U_2)\rightarrow \varphi_2(U_1\cap U_2)
			\end{equation*}
			Da die Karten Homöomorphismen sind, sind auch alle Kartenwechsel Homöomorphismen.
		\end{defn} 
		\begin{defn}[Differenzierbare Mannigfaltigkeit]\hfill
			\begin{itemize}
				\item Zwei Karten heißen verträglich, wenn ihr Kartenwechsel ein (glatter) Diffeomorphismus ist (das heißt, der Kartenwechsel und sein Inverses sind glatt).
				\item Ein Atlas heißt differenzierbar, wenn alle seine Karten paarweise verträglich sind.
				\item Ein differenzierbarer Atlas heißt maximal, wenn es keine Karte gibt, die mit allen Karten des Atlanten verträglich und nicht in ihm enthalten ist.
				\item Eine differenzierbare Struktur auf einer topologischen Mannigfaltigkeit $M$ ist ein maximaler differenzierbarer Atlas $\mathcal{A}$ und das Tupel $(M,\mathcal{A})$ heißt differenzierbare Mannigfaltigkeit.
			\end{itemize}
		\end{defn}
	\section{Differenzierbare Abbildungen}
		Seien von nun an $M,N$ differenzierbare Mannigfaltigkeiten.
		\begin{defn}[Differenzierbarkeit]
			Eine Abbildung $f:M\rightarrow N$ zwischen differenzierbaren Mannigfaltigkeiten heißt differenzierbar, wenn für alle Punkte $p\in M$ Karten $(\varphi,U)$ um  $p$ und $(\psi,V)$ um $f(p)$ existieren, sodass $f(U)\subset V$ und die folgende Abbildung glatt ist:
			\begin{equation}
				\psi\circ f \circ \varphi^{-1}:\varphi(U)\rightarrow \psi(V)
			\end{equation}
			Diese Definition ist unabhängig von der Wahl der Karte, da Kartenwechsel glatt sind. Äquivalent kann daher auch gefordert werden, dass die obige Abbildung für alle Karten glatt ist.\\
			Wir schreiben $\sC(M,N)$ für die Menge der differenzierbaren Abbildungen zwischen Mannigfaltigkeiten $M,N$.
		\end{defn}
		\begin{nota}
			$\sm:=\sC(M,\R)$ ist die Menge der (differenzierbaren) Funktionen auf $M$.
		\end{nota}
		\begin{defn}
			Eine Abbildung $f\in\sC(M,N)$ heißt Diffeomorphismus, wenn $f$ bijektiv ist und $f^{-1}\in\sC(N,M)$ gilt.
		\end{defn}
%%%%%%%%%%%%%%%%%%%%%%%%%%%%%%%%%%%%%%%%%%%%%%%%%%%%%%%%%%%
\chapter{Der Tangentialraum, das Differential, Untermannigfaltigkeiten}


%%%%%%%%%%%%%%%%%%%%%%%%%%%%%%%%%%%%%%%%%%%%%%%%%%%%%%%%%%%
\chapter{Vektorbündel}


%%%%%%%%%%%%%%%%%%%%%%%%%%%%%%%%%%%%%%%%%%%%%%%%%%%%%%%%%%%
\chapter{Vektorfelder}


%%%%%%%%%%%%%%%%%%%%%%%%%%%%%%%%%%%%%%%%%%%%%%%%%%%%%%%%%%%
\chapter{Lie-Gruppen}


%%%%%%%%%%%%%%%%%%%%%%%%%%%%%%%%%%%%%%%%%%%%%%%%%%%%%%%%%%%
\chapter{Tensorfelder}


%%%%%%%%%%%%%%%%%%%%%%%%%%%%%%%%%%%%%%%%%%%%%%%%%%%%%%%%%%%
\chapter{Differentialformen}


%%%%%%%%%%%%%%%%%%%%%%%%%%%%%%%%%%%%%%%%%%%%%%%%%%%%%%%%%%%
\chapter{Kovariante Ableitungen}


%%%%%%%%%%%%%%%%%%%%%%%%%%%%%%%%%%%%%%%%%%%%%%%%%%%%%%%%%%%
\chapter{Riemann'sche Geometrie}


%%%%%%%%%%%%%%%%%%%%%%%%%%%%%%%%%%%%%%%%%%%%%%%%%%%%%%%%%%%
\chapter{Integration auf Mannigfaltigkeiten}


\end{document}