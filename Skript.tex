% !TeX spellcheck = de_DE
\documentclass[a4paper]{scrreprt}
\usepackage[utf8]{inputenc}
\usepackage[T1]{fontenc}
\usepackage{lmodern}
\usepackage[sc]{mathpazo}
\linespread{1.05}
\usepackage{exscale} % To scale mathematical symbols correctly while using T1
\usepackage{amsmath,amsthm,amssymb,dsfont}
\usepackage{tikz}
	\usetikzlibrary{matrix}
\usepackage[ngerman]{babel}
\usepackage{enumitem}
\usepackage{microtype} % Microtypography!
\usepackage{epigraph}
\usepackage{hyperref}
\usepackage{todonotes}

\setlength{\epigraphwidth}{0.6\textwidth}

\numberwithin{equation}{chapter}

\newcommand{\D}{\mathrm{d}}
\newcommand{\DD}{\mathrm{D}}
\newcommand{\e}{\mathrm{e}}
\newcommand{\diff}{:\Longleftrightarrow}
\DeclareMathOperator{\id}{id}
\DeclareMathOperator{\Diff}{Diff}
\DeclareMathOperator{\GL}{GL}
\DeclareMathOperator{\pr}{pr}

\newcommand{\R}{\mathbb{R}}
\newcommand{\sC}{\mathcal{C}^{\infty}}
\newcommand{\sm}{\mathcal{F}(M)}
\newcommand{\vf}{\mathfrak{X}(M)}
\newcommand{\tril}{\vartriangleleft}
\newcommand{\trir}{\vartriangleright}

\theoremstyle{definition}
\newtheorem{defn}{Definition}[section]
\newtheorem{lemma}[defn]{Lemma}
\newtheorem{prop}[defn]{Proposition}
\newtheorem{satz}[defn]{Satz}
\newtheorem{kor}[defn]{Korollar}
\newtheorem{bem}[defn]{Bemerkung}
\newtheorem{bsp}[defn]{Beispiel}
\newtheorem{nota}[defn]{Notation}

\newcommand{\bewUeb}{\begin{proof}Übung.\end{proof}}

% Kommentare
\newcommand{\kommP}[2][noinline]{\todo[#1,color=green!40]{#2}}
\newcommand{\kommB}[2][noinline]{\todo[#1,color=blue!20]{#2}}

\title{Einführung in die Differentialgeometrie}
\subtitle{Kurs auf der CdE-WinterAkademie 2018/19}
\author{Benjamin Haake, Philip Schwartz}
\date{November 2018}

\begin{document}

\maketitle

%%%%%%%%%%%%%%%%%%%%%%%%%%%%%%%%%%%%%%%%%%%%%%%%%%%%%%%%%%%
\setcounter{chapter}{-1}
\chapter{Einleitung}
\epigraph{Differentialgeometrie ist die Lehre von Eigenschaften, die invariant unter Notationswechsel sind.}{\textsc{Altes chinesisches Sprichwort}}

Verallgemeinerung von Kurven, Flächen und so. Extrem wichtig innerhalb der Mathematik und auch in quasi allen Anwendungsgebieten, insb. der theoretischen Physik (ART, Eichtheorien, alles!).

\section{Bekannte Konzepte, Notationen etc.}

\subsection{Lineare Algebra}

\begin{nota}
	Für einen Vektorraum $V$ über einem Körper $K$ bezeichnet \[V^* := \mathrm{Hom}(V,K) = \{f\colon V \to K : f \text{ linear}\}\] den Dualraum von $V$.
\end{nota}

\subsection{Mehrdimensionale Analysis}
\begin{nota}
	Punkte im $\mathbb R^n$ schreiben wir als $a = (a^1, \dots, a^n)$. Die Vektoren der Standardbasis von $\mathbb R^n$ schreiben wir als $\e_1, \dots, \e_n \in \mathbb R^n$, also \[\e_i = (0,\dots,\underset{i}{1},\dots,0).\]

	Sei $U \subset \mathbb R^n$ eine offene Menge und $f\colon U \to \mathbb R$ eine differenzierbare Abbildung. Die partielle Ableitung von $f$ nach der $i$-ten Komponente schreiben wir als $\partial_i f$, d.\,h. es ist
	\[(\partial_i f)(a) = \left.\frac{\D f(a + h \e_i)}{\D h}\right|_{h=0} = \lim_{h\to 0} \frac{f(a + h \e_i) - f(a)}{h}.\]
\end{nota}

\begin{satz}[Mehrdimensionale Kettenregel]
	Seien $U \subset \mathbb R^l, V \subset \mathbb R^m, W \subset \mathbb R^n$ offene Mengen und $f\colon U\to V, g\colon V \to W$ differenzierbare Abbildungen. Dann ist auch $g\circ f\colon U \to W$ differenzierbar, und für die Differentiale gilt
	\[\DD(g\circ f)|_a = \DD g|_{f(a)} \circ \DD f|_a\]
	für $a \in U$. Für die partiellen Ableitungen der Komponenten gilt also
	\[\partial_i(g\circ f) = \sum_{j = 1}^m ((\partial_j g)\circ f) \cdot \partial_i f^j,\]
	d.\,h.
	\[(\partial_i(g\circ f))(a) = \sum_{j = 1}^m (\partial_j g)(f(a)) \cdot (\partial_i f^j)(a).\]
\end{satz}


%%%%%%%%%%%%%%%%%%%%%%%%%%%%%%%%%%%%%%%%%%%%%%%%%%%%%%%%%%%
\chapter{Mannigfaltigkeiten}
	\section{Topologie}
		\begin{defn}[Topologischer Raum]
			Ein Tupel $(X,\mathcal{T})$ bestehend aus einer Menge $X$ und einer Familie von Teilmengen $\mathcal{T}$ heißt \emph{topologischer Raum}, wenn folgende Eigenschaften erfüllt sind:
			\begin{enumerate}[label=$T$\arabic*]
				\item $\emptyset, x\in \mathcal{T}$
				\item $\forall U_1, U_2\in \mathcal{T} \text{ gilt }U_1\cap U_2\in\mathcal{T}$
				\item Für jede Familie $\lbrace U_i\rbrace_{i\in I}\subset \mathcal{T}$ gilt $\bigcup_{i\in I}U_i\in\mathcal{T}$
			\end{enumerate}
			Die Familie $\mathcal{T}$ heißt \emph{Topologie} auf der Menge $X$, ihre Elemente heißen \emph{offene Mengen}. Eine Menge heißt \emph{abgeschlossen}, wenn ihr Komplement offen ist.
		\end{defn}
		\begin{defn}[Basis einer Topologie]
			Eine Familie offener Mengen $\lbrace B_i\rbrace_{i\in I}\subset \mathcal{T}$ heißt \emph{Basis} eines topologischen Raums $(X,\mathcal{T})$, wenn sich jede offene Menge als Vereinigung von Elementen der Basis schreiben lässt, das heißt:
			\begin{equation*}
				\forall U\in\mathcal{T}\exists \lbrace B_i\rbrace_{i\in J}\subset \mathcal{T}: U=\bigcup_{i\in J}B_i
			\end{equation*}
		\end{defn}
		\begin{bsp}
			Ist $(X,d)$ ein metrischer Raum, so bilden die bzgl. der Metrik offenen Mengen eine Topologie mit den $\varepsilon$-Bällen $\lbrace B_{\varepsilon}(x)\vert \; x\in X, \varepsilon >0\rbrace$ als Basis.
		\end{bsp}
		\begin{defn}[Zweitabzählbarkeit]
			Ein topologischer Raum $(X,\mathcal{T})$ heißt \emph{zweitabzählbar}, wenn für die Topologie $\mathcal{T}$ eine abzählbare Basis existiert. 
		\end{defn}
		\begin{defn}[Kompaktheit]
			Eine Teilmenge $K\subset X$ heißt \emph{kompakt}, wenn jede offene Überdeckung (das heißt jede Familie offener Mengen $\lbrace U_i\rbrace_{i\in I}$ mit $K\subset \bigcup_{i\in I}U_i$) eine endliche Teilüberdeckung hat (also endlich viele $U_1,\ldots,U_n$ existieren sodass schon $K\subset \bigcup_{i=1}^n U_i$ gilt).
		\end{defn}
		\begin{defn}[Hausdorffraum]
			Ein topologischer Raum $(X,\mathcal{T})$ heißt \emph{hausdorffsch} oder \emph{Hausdorffraum}, falls für je zwei Punkte $x,y\in X, x\neq y$ offene Umgebungen ${U_x,U_y\in\mathcal{T}}$, ${x\in U_x}, {y\in U_y}$ existieren sodass $U_x\cap U_y=\emptyset$. Zwei Punkte können also durch offene Mengen \glqq getrennt\grqq\ werden.
		\end{defn}
		\begin{lemma}
			In einem Hausdorffraum sind kompakte Mengen abgeschlossen.
		\end{lemma}
		\begin{bsp}
			Der euklidische Raum $(\R^n,d)$ ist mit der metrischen Topologie ein topologischer Raum. Er ist hausdorffsch und zweitabzählbar.
		\end{bsp}
		\begin{defn}[Stetige Abbildung]
			Eine Abbildung $f\colon (X,\mathcal{T}_X)\rightarrow (Y,\mathcal{T}_Y)$ zwischen topologischen Räumen heißt \emph{stetig}, wenn Urbilder offener Mengen offen sind:
			\begin{equation*}
				U\in\mathcal{T}_Y \Rightarrow f^{-1}(U)\in\mathcal{T}_X
			\end{equation*}
			Äquivalent dazu ist, dass Urbilder abgeschlossener Mengen abgeschlossen sind (da Komplementbildung mit dem Urbild vertauscht).
		\end{defn}
		Die Topologien werden im Folgenden in der Notation unterdrückt.
		\begin{lemma}
			Die Verkettung stetiger Abbildungen ist stetig.
		\end{lemma}
		\begin{lemma}
			Das Bild einer kompakten Menge unter einer stetigen Abbildung ist kompakt:
			\begin{equation*}
				f\colon X\rightarrow Y \text{ stetig }, K\subset X \text{ kompakt }\Rightarrow f(K)\subset Y \text{ kompakt}
			\end{equation*}
		\end{lemma}
		\begin{defn}[Homöomorphismus]
			Eine stetige Abbildung $f\colon X\rightarrow Y$ heißt \emph{Homöomorphismus}, wenn sie bijektiv und ihre Umkehrabbildung stetig ist.
		\end{defn}
	\section{Mannigfaltigkeiten}
		\begin{defn}
			Ein topologischer Raum $(M,\mathcal{T})$ heißt \emph{topologische Mannigfaltigkeit}, wenn er hausdorffsch, zweitabzählbar und lokal euklidisch ist, das heißt:
			Es existiert eine natürliche Zahl $n\in \mathbb{N}$ (Dimension), sodass für alle Punkte $x\in M$ offene Umgebungen $U\subset M, V\subset \R^n$ mit $x\in U $ und ein Homöomorphismus $ \varphi\colon U\rightarrow V$ existieren.

			Die Dimension der Mannigfaltigkeit ist wohldefiniert (Beachte, dass $n$ unabhängig von der Wahl des Punktes existieren muss).

			Das Tupel $(\varphi,U)$ heißt \emph{Karte} (wobei manchmal die Menge $U$ unterdrückt wird). Eine Familie von Karten, die die Mannigfaltigkeit überdecken, heißt \emph{Atlas} $\mathcal{A}$.
		\end{defn}
		\begin{bsp}\hfill 
			\begin{enumerate}
				\item Der euklidische Raum $\R^n$ ist eine $n$-dimensionale topologische Mannigfaltigkeit, beispielsweise mit dem Atlas $\lbrace (id_{\R^n},\R^n)\rbrace$. 
				\item Das kartesische Produkt zweier topologischer Mannigfaltigkeiten (der Dimensionen $n$ und $m$) ist eine topologische Mannigfaltigkeit (der Dimension $n + m$). Die Produkte offener Mengen bilden dabei eine Basis der Topologie des Produktes.
				\item Für topologische Mannigfaltigkeiten gleicher Dimension ist die disjunkte Vereinigung eine topologische Mannigfaltigkeit (\emph{topologische Summe}).
			\end{enumerate}
		\end{bsp}
		\begin{defn}
			Für zwei Karten $\varphi_i\colon U_i\rightarrow V_i, i\in\lbrace 1,2 \rbrace$ heißt die folgende Abbildung \emph{Kartenwechsel}:
			\begin{equation*}
				\varphi_2\circ\varphi_1^{-1}\vert_{\varphi_1(U_1\cap U_2)}\colon \varphi_1(U_1\cap U_2)\rightarrow \varphi_2(U_1\cap U_2)
			\end{equation*}
			Da die Karten Homöomorphismen sind, sind auch alle Kartenwechsel Homöomorphismen.
		\end{defn} 
		\begin{defn}[Differenzierbare Mannigfaltigkeit]\hfill
			\begin{itemize}
				\item Zwei Karten heißen \emph{verträglich}, wenn ihr Kartenwechsel ein (glatter) Diffeomorphismus ist (das heißt, der Kartenwechsel und sein Inverses sind glatt).
				\item Ein Atlas heißt \emph{differenzierbar}, wenn alle seine Karten paarweise verträglich sind.
				\item Ein differenzierbarer Atlas heißt \emph{maximal}, wenn es keine Karte gibt, die mit allen Karten des Atlanten verträglich und nicht in ihm enthalten ist.
				\item Eine \emph{differenzierbare Struktur} auf einer topologischen Mannigfaltigkeit $M$ ist ein maximaler differenzierbarer Atlas $\mathcal{A}$ und das Tupel $(M,\mathcal{A})$ heißt \emph{differenzierbare Mannigfaltigkeit}.
				\kommP{Sollten erwähnen, dass, wenn Mf. gegeben, Karten immer als aus der Struktur gemeint sind.}
			\end{itemize}
		\end{defn}
		\begin{bsp}\hfill 
			\begin{enumerate}
				\item Der euklidische Raum $\R^n$ ist eine $n$-dimensionale differenzierbare Mannigfaltigkeit, der oben genannte Atlas muss dazu aber erweitert werden!
				\kommP{Erwähnen, dass jeder dfb. Atlas einen eindeutige dfb. Struktur induziert, mit der er verträglich ist? Übungsaufgabe?}
				\item Reelle, endlich dimensionale Vektorräume sind differenzierbare Mannigfaltigkeiten.
				%\item Das kartesische Produkt zweier differenzierbarer Mannigfaltigkeiten (der Dimensionen $n$ und $m$) ist eine differenzierbare Mannigfaltigkeit (der Dimension $n+m$).
				\item Für differenzierbare Mannigfaltigkeiten gleicher Dimension ist die disjunkte Vereinigung eine differenzierbare Mannigfaltigkeit (\emph{topologische Summe}).
			\end{enumerate}
		\end{bsp}
	\section{Differenzierbare Abbildungen}
		Seien von nun an $M,N$ differenzierbare Mannigfaltigkeiten.
		\begin{defn}[Differenzierbarkeit]
			Eine Abbildung $f\colon M\rightarrow N$ zwischen differenzierbaren Mannigfaltigkeiten heißt \emph{glatt}, wenn für alle Punkte $p\in M$ Karten $(\varphi,U)$ um  $p$ und $(\psi,V)$ um $f(p)$ existieren, sodass $f(U)\subset V$ und die folgende Abbildung glatt ist:
			\kommP{Evtl. $C^k$ definieren?}
			\begin{equation}
				\psi\circ f \circ \varphi^{-1}\colon \varphi(U)\rightarrow \psi(V)
			\end{equation}
			Diese Definition ist unabhängig von der Wahl der Karte, da Kartenwechsel glatt sind. Äquivalent kann daher auch gefordert werden, dass die obige Abbildung für alle Karten glatt ist.

			Wir schreiben $\sC(M,N)$ für die Menge der glatten Abbildungen.
		\end{defn}
		\begin{nota}
			$\sm:=\sC(M,\R)$ ist die Menge der (glatten) Funktionen auf $M$.
		\end{nota}
		\begin{defn}
			Eine Abbildung $f\in\sC(M,N)$ heißt Diffeomorphismus, wenn $f$ bijektiv ist und $f^{-1}\in\sC(N,M)$ gilt.

			Die Menge der Diffeomorphismen $\Diff(M):=\sC(M,M)$ bildet eine Gruppe.
		\end{defn}

%%%%%%%%%%%%%%%%%%%%%%%%%%%%%%%%%%%%%%%%%%%%%%%%%%%%%%%%%%%
\chapter{Der Tangentialraum, das Differential, Untermannigfaltigkeiten}

%%%%%%%%%%%%%%%%%%%%%%%%%%%%%%%%%%%%%%%%%%%%%%%%%%%%%%%%%%%
\section{Der Tangentialraum}
Idee: Was sind \glqq Richtungen\grqq\ auf einer Mannigfaltigkeit?

Anschaulich: Wenn Mf. eingebettet im $\mathbb R^n$, können wir uns eine Tangentialebene vorstellen (Bild: $S^2$ in $\mathbb R^3$). Also: Mögliche Richtungen = Ableitungen von Kurven. Aber abstrakt?

Müssen intrinsisch darüber reden können. Dabei hilft die Beobachtung, dass man Richtungen dazu benutzt, zu beschreiben, wie sich Dinge bei Bewegung in diese Richtung ändern. Also: Richtung = Richtungsableitung! \todo{Ableitung entlang von Kurven im $\mathbb R^n$ als Motivation.} Wir werden deshalb \glqq Richtungen\grqq\ als Ableitungsoperatoren formalisieren.

Idee: Ableitungen erfüllen eine Produktregel und wirken \glqq lokal\grqq.

Sei im Folgenden $M$ eine differenzierbare Mannigfaltigkeit, $\dim M = n$.

\begin{defn}[Funktionskeime]
	Sei $p \in M$. Auf der Menge $X = \{f \in \sC(U) : U \subset M \text{ offene Umgebung von } p\}$ betrachten wir die Äquivalenzrelation $\sim$, die gegeben ist durch
	\[f \sim g \diff \exists \text{ offene Umgebung } V \text{ von } p \text{ mit } f = g \text{ auf } V.\]
	Die Äquivalenzklassen bzgl. dieser Relation heißen glatte \emph{Funktionskeime} an $M$ im Punkt $p$. Zwei um $p$ definierte Funktionen definieren also genau dann denselben Keim, wenn sie in einer Umgebung von $p$ übereinstimmen.

	Den Raum der Funktionskeime an $M$ in $p$ schreiben wir als $\sC_p(M)$. Den Keim zu einer Funktion $f$ schreiben wir als $[f]$. Wenn keine Missverständnisse entstehen können, schreiben wir stattdessen manchmal auch $f$, um uns Notationsaufwand zu ersparen.
\end{defn}

Zu einem Funktionskeim $[f] \in \sC_p(M)$ ist der Funktionswert $[f](p) := f(p)$ wohldefiniert (warum?).

\begin{prop}
	$\sC_p(M)$ ist mit über die Repräsentanten definierter Addition, Skalarmultiplikation und Multiplikation eine $\mathbb R$-Algebra. \bewUeb
\end{prop}

\begin{defn}
	Eine \emph{Derivation} von $\sC_p(M)$ ist eine lineare Abbildung $v\colon \sC_p(M) \to \mathbb R$, die die \glqq Produktregel\grqq\ \[v(fg) = v(f) g(p) + f(p) v(g)\] erfüllt. Den Raum der Derivationen von $\sC_p(M)$ nennen wir den \emph{Tangentialraum} $T_pM$ an $M$ im Punkt $p$; die Elemente von $T_pM$ heißen auch \emph{Tangentialvektoren} in $p$.
\end{defn}

$T_pM$ ist tatsächlich ein Untervektorraum von $(\sC_p(M))^*$ (Übung).

\begin{lemma}[Derivationen verschwinden auf Konstanten]
	Für alle $v \in T_pM$ und $c \in \mathbb R$ ist $v(c) = 0$ (dabei fassen wir $c$ als konstante Funktion bzw. den davon induzierten Funktionskeim auf).
	
	\begin{proof}
		Es ist $v(1) = v(1\cdot 1) = v(1) 1(p) + 1(p) v(1) = 2 v(1)$, also $v(1) = 0$. Da $v$ linear ist, folgt die Behauptung.
	\end{proof}
\end{lemma}

\begin{bsp} \label{bsp:kurve_geschw}
	Sei $I\subset\mathbb R$ offen und $\gamma\colon I \to M$ eine glatte Kurve mit $\gamma(s) = p$ für ein $s \in I$. Wir definieren den Tangentialvektor $\gamma'(s) \in T_pM$ durch \[(\gamma'(s))(f) := (f\circ\gamma)'(s),\] wobei wir auf der rechten Seite die Ableitung der Funktion $f\circ\gamma \colon I \to \mathbb R, I \subset \mathbb R$ gebildet haben.

	$\gamma'(s)$ leitet also die Funktionskeime, auf die es angewendet wird, entlang der Kurve ab; wir können $\gamma'(s)$ also als \glqq Geschwindigkeitsvektor\grqq\ der Kurve auffassen.

	$\gamma'(s)$ ist wohldefiniert und tatsächlich eine Derivation: Wenn $[f] = [g]$ gilt, dann stimmen $f$ und $g$ in einer Umgebung von $p$ überein, und damit stimmen $f\circ\gamma$ und $g\circ\gamma$ in einer Umgebung von $s$ überein und haben insbesondere dieselbe Ableitung an der Stelle $s$. Die Linearität von $\gamma'(s)$ folgt aus der Linearität der Ableitung, und die Derivations-Produktregel folgt aus der Produktregel für Ableitungen.
\end{bsp}

Später werden wir sehen, dass sich tatsächlich \emph{jeder} Tangentialvektor als Ableitung einer Kurve schreiben lässt.

%%%%%%%%%%%%%%%%%%%%%%%%%%%%%%%%%%%%%%%%%%%%%%%%%%%%%%%%%%%
\section{Das Differential}

Mithilfe des Tangentialraumbegriffs können wir nun nicht nur entscheiden, ob Abbildungen differenzierbar sind, sondern auch tatsächlich eine Art Ableitungsbegriff definieren:
\begin{defn}
	Seien $M,N$ differenzierbare Mannigfaltigkeiten und $f\colon M \to N$ eine glatte Abbildung. Das \emph{Differential} von $f$ im Punkt $p\in M$ ist die lineare Abbildung
	\[\left.\DD f\right|_p \colon T_pM \to T_{f(p)}N\]
	definiert durch
	\[\left(\left.\DD f\right|_p(v)\right)(\alpha) = v (\alpha\circ f)\]
	für $v\in T_pM, \alpha \in \sC_{f(p)}(N)$.
\end{defn}
\begin{prop}
	$\left.\DD f\right|_p(v)$ ist tatsächlich wohldefiniert und eine Derivation von $\sC_{f(p)}(N)$, und $\left.\DD f\right|_p$ ist linear. \bewUeb
\end{prop}

\begin{satz}[Kettenregel]
	Seien $f\colon M \to N, g\colon N \to L$ glatte Abbildungen zwischen differenzierbaren Mannigfaltigkeiten, und $p \in M$. Dann gilt \[\left.\DD(g\circ f)\right|_p = \left.\DD g\right|_{f(p)} \circ \left.\DD f\right|_p.\]

	\begin{proof}
		Sei $v \in T_pM$. Für jedes $\alpha \in \sC_{g(f(p))}(L)$ ist
		\begin{align*}
			\left(\left.\DD(g\circ f)\right|_p(v)\right)(\alpha) &= v(\alpha\circ (g \circ f))\\
			&= v((\alpha\circ g) \circ f)\\
			&= \left(\left.\DD f\right|_p(v)\right) (\alpha \circ g)\\
			&= \left(\left.\DD g\right|_{f(p)} \left(\left.\DD f\right|_p(v)\right)\right) (\alpha),
		\end{align*}
		also gilt $\left.\DD(g\circ f)\right|_p(v) = \left.\DD g\right|_{f(p)}\left(\left.\DD f\right|_p(v)\right)$.
	\end{proof}
\end{satz}

\begin{lemma} \label{lemma:differential_diffeo}
	\begin{enumerate}[label=(\alph*)]
		\item $\left.\DD(\id_M)\right|_p = \id_{T_pM}$
		\item Ist $f\colon M \to N$ ein Diffeomorphismus, so ist $\left.\DD f\right|_p$ für jedes $p\in M$ ein Isomorphismus, und es gilt $\left.\DD f^{-1}\right|_{f(p)} = \left(\left.\DD f\right|_p\right)^{-1}$.
	\end{enumerate}
	\bewUeb
\end{lemma}

\begin{prop}
	Sei $M$ eine differenzierbare Mannigfaltigkeit und $U\subset M$ eine offene Teilmenge. Für jedes $p\in U$ ist $T_pU$ kanonisch isomorph zu $T_pM$ vermöge $\left.\DD i\right|_p$, wobei $i\colon U \to M, i(q) := q$ die Inklusionsabbildung ist.

	\begin{proof}
		Sei $p \in U$. $\left.\DD i\right|_p$ ist definiert über $\left(\left.\DD i\right|_p(v)\right)(\alpha) = v(\alpha\circ i)$ für $v \in T_pU$ und $\alpha \in \sC_p(M)$; d.\,h. $\left.\DD i\right|_p(v)$ ist einfach $v$, nur aufgefasst als Derivation, die Funktionskeime an $M$ ableitet, indem man diese als Funktionskeime an $U$ auffasst. Da Funktionskeime an $U$ in $p$ und Funktionskeime an $M$ in $p$ aber einander entsprechen, ist damit klar, dass $\left.\DD i\right|_p$ ein Isomorphismus ist.

		Etwas genauer: Jeder Funktionskeim $\tilde\alpha \in \sC_p(U)$ kann auch als Keim $\alpha \in \sC_p(M)$ aufgefasst werden, und umgekehrt: Jede offene Umgebung von $p$ in $U$ ist auch offen in $M$, und jede offene Umgebung von $p$ in $M$ enthält eine offene Umgebung von $p$ in $U$. Da bei der Definition von Funktionskeimen um $p$ Funktionen äquivalent sind, wenn sie auf einer offenen Umgebung von $p$ übereinstimmen, spielt es also keine Rolle, ob wir in $U$ oder in $M$ sind (\glqq es wird eh alles äquivalent gemacht\grqq)\footnote{
			Als Element von $\sC_p(M)$ besteht der Keim / die Äquivalenzklasse potentiell aus mehr Funktionen, da auch noch Funktionen dazukommen, die auf \glqq größeren\grqq\ Umgebungen von $p$ definiert sind.
		}.

		Diese Abbildung $\sC_p(M) \ni \alpha \mapsto \tilde\alpha \in \sC_p(U)$ ist also für $[f]_{p,M} \in \sC_p(M)$, $f$ definiert auf $V$, gegeben durch $[f]_{p,M} \mapsto \left[\left.f\right|_{U\cap V}\right]_{p,U} \in \sC_p(U)$, was man auch schreiben kann als $[f]_{p,M} \mapsto \left[\left.f\circ i\right|_{U\cap V}\right]_{p,U}$. D.\,h. das oben beschriebene Auffassen von $\alpha$ als Keim auf der kleineren Mannigfaltigkeit $U$ ist die Abbildung $\sC_p(M) \ni \alpha \mapsto \alpha\circ i \in \sC_p(U)$. Wie oben beschrieben lässt sich das umkehren, ist also ein Isomorphismus. Damit ist aber auch $\left.\DD i\right|_p$ ein Isomorphismus, denn es ist definiert als $\left(\left.\DD i\right|_p(v)\right)(\alpha) = v(\alpha\circ i)$.
	\end{proof}
\end{prop}

Da $\left.\DD i\right|_p(v)$ einfach dieselbe Derivation wie $v$ ist (sie unterscheiden sich nur darin, ob man die Keime, auf die sie wirkt, als Keime an $M$ oder an $U$ betrachtet), werden wir im Folgenden $T_pU$ und $T_pM$ (meistens) miteinander identifizieren.

%%%%%%%%%%%%%%%%%%%%%%%%%%%%%%%%%%%%%%%%%%%%%%%%%%%%%%%%%%%
\section{Der Tangentialraum eines Vektorraums}

Betrachten wir einen endlich-dimensionalen Vektorraum $V$, so sagt uns die Intuition, dass die Richtungen, in die wir ableiten können, einfach die Richtungen des Vektorraums sein sollten. Wir erwarten also einen kanonischen Isomorphismus $T_pV \cong V$ für alle $p \in V$. Sei im Folgenden $V$ ein endlich-dimensionaler Vektorraum und $p \in V$ ein fester Punkt.
\begin{defn}
	Für $v \in V$ definieren wir den Tangentialvektor $\left.\partial_v\right|_p \in T_pV$ durch
	\[\left.\partial_v\right|_p(f) := (\partial_v f)(p) = \left.\frac{\D f(p + h v)}{\D h}\right|_{h=0} = \lim_{h\to 0} \frac{f(p + h v) - f(p)}{h}.\]
\end{defn}
$\left.\partial_v\right|_p$ ist also einfach die Richtungsableitung in Richtung von $v$ am Punkt $p$.
\begin{prop}
	$\left.\partial_v\right|_p$ ist tatsächlich eine Derivation von $\sC_p(V)$.
	
	\begin{proof}
		$\left.\partial_v\right|_p$ ist wohldefiniert, denn wenn $[f] = [g]$ gilt, dann stimmen $f$ und $g$ in einer Umgebung von $p$ überein und haben insbesondere dieselben Ableitungen an der Stelle $p$. Dass Richtungsableitungen linear sind und die Produktregel erfüllen, ist bekannt\footnote{Oder folgt schnell aus den entsprechenden Eigenschaften für Ableitungen von Funktionen $\mathbb R \to \mathbb R$.}.
	\end{proof}
\end{prop}

\begin{satz} \label{satz:tangt_vr}
	Die Abbildung \[V \to T_pV, v \mapsto \left.\partial_v\right|_p\] ist ein linearer Isomorphismus.
	
	\begin{proof}\let\qed\relax
		Linearität ist einfach zu zeigen (Übung). Um Surjektivität und Injektivität zu zeigen, fixieren wir eine Basis $\{b_1,\dots,b_n\}$ von $V$. Sei $\{\theta^1, \dots, \theta^n\}$ die dazu duale Basis von $V^*$, also für $i = 1, \dots, n$ jeweils $\theta^i\colon V \to \mathbb R$ die lineare Abbildung definiert durch $\theta^i(b_j) = \delta^i_j$ und lineare Fortsetzung.

		Für beliebiges $v = \sum_{i=1}^n v^i b_i \in V$ ist $\left.\partial_v\right|_p(\theta^i) = \left.\frac{\D\theta^i(p + h v)}{\D h}\right|_{h=0} = \left.\frac{\D(\theta^i(p) + h v^i)}{\D h}\right|_{h=0} = v^i$ für alle $i$. Falls $\left.\partial_v\right|_p = 0$ gilt, ist also $v^i = \left.\partial_v\right|_p(\theta^i) = 0(\theta^i) = 0$, also $v=0$. Das zeigt, dass der Kern der betrachteten Abbildung nur die 0 enthält, also deren Injektivität. Um die Surjektivität zu zeigen, brauchen wir einen kleinen Trick.
	\end{proof}
\end{satz}

\begin{lemma}[Hadamard-Lemma] \label{lemma:Hadamard}
	Seien $U \subset \mathbb R^n$ eine offene Menge, die sternförmig um $x_0$ ist, und $f \in \sC(U)$. Dann gibt es glatte Funktionen $f_i \in \sC(U), i = 1,\dots,n$, sodass \[f(x) = f(x_0) + \sum_{i=1}^n f_i(x) (x^i - x_0^i).\]
	
	\begin{proof}
		Für $x \in U$ setze $y := x - x_0$. Dann ist
		\begin{align*}
		f(x) - f(x_0) &= f(x_0 + 1\cdot y) - f(x_0 + 0\cdot y)\\
		&= \int_0^1 \left(\frac{\D}{\D t} f(x_0 + ty)\right) \D t\\
		\text{(Kettenregel)} \quad &= \int_0^1 \sum_{i=1}^n (\partial_i f)(x_0 + ty) y^i \D t\\
		&= \sum_{i=1}^n \left(\int_0^1 (\partial_i f)(x_0 + tx - tx_0) \D t\right) (x^i - x_0^i).
		\end{align*}
		Mit \[f_i(x) := \int_0^1 (\partial_i f)(x_0 + tx - tx_0) \D t\] folgt also die Behauptung (die $f_i$ sind glatt wegen Kettenregel und Sätzen über das Differenzieren unter dem Integral).
	\end{proof}
\end{lemma}

\begin{kor}[Hadamard-Lemma auf Vektorräumen]
	Sei $\theta = (\theta^1, \dots, \theta^n) \colon V \to \mathbb R^n$ ein linearer Isomorphismus und sei $\alpha \in \sC_p(V)$. Dann gibt es $\alpha_i \in \sC_p(V), i = 1,\dots, n$ mit \[\alpha = \alpha(p) + \sum_{i=1}^n \alpha_i \cdot (\theta^i - \theta^i(p)).\]
	
	\begin{proof}
		Das ist einfach das Hadamard-Lemma \ref{lemma:Hadamard} angewendet auf unseren Vektorraum $V$, der über $\theta$ mit $\mathbb R^n$ identifiziert wurde.
	\end{proof}
\end{kor}

\begin{proof}[Zurück zum Beweis von Satz \ref{satz:tangt_vr}]
	Mit dem Hadamard-Lemma können wir jetzt auch zeigen, dass die Abbildung $v \mapsto \left.\partial_v\right|_p$ surjektiv ist. Seien $w \in T_pV$ und $\alpha \in \sC_p(V)$ beliebig, und seien $\alpha_i \in \sC_p(V)$ zu $\alpha$ mit dem Hadamard-Lemma gewählt. Dann ist
	\begin{align}\begin{split} \label{eq:pf_tangt_vr}
		w(\alpha) &= w\left(\alpha(p) + \sum_{i=1}^n \alpha_i \cdot (\theta^i - \theta^i(p))\right)\\
		\text{(Linearität, $w(\text{const.}) = 0$)} \quad &= \sum_{i=1}^n w(\alpha_i \cdot (\theta^i - \theta^i(p)))\\
		&= \sum_{i=1}^n \left( w(\alpha_i) \cdot (\theta^i(p) - \theta^i(p)) + \alpha_i(p) \cdot w(\theta^i) \right)\\
		&= \sum_{i=1}^n \alpha_i(p) \cdot w(\theta^i).
	\end{split}\end{align}
	Insbesondere gilt damit auch $\left.\partial_v\right|_p (\alpha) = \sum_{j=1}^n \alpha_i(p) \cdot \left.\partial_v\right|_p (\theta^i) = \sum_{j=1}^n \alpha_i(p) \cdot v^i$ für beliebige Vektoren $v = \sum_{i=1}^n v^i b_i$. Wenn wir also $v := \sum_{i=1}^n w(\theta^i) b_i$ setzen, ist $w(\alpha) = \left.\partial_v\right|_p (\alpha)$.

	Da $\alpha$ beliebig war, gilt also $w = \left.\partial_v\right|_p$. Wir können also jedes $w \in T_pV$ als Richtungsableitung in Richtung eines Vektors $v\in V$ schreiben.
\end{proof}

Der Tangentialraum $T_pV$ an einen Vektorraum $V$ ist also in kanonischer Weise zu $V$ selbst isomorph; ein Vektor $v \in V$ wird dabei mit der Richtungsableitung $\left.\partial_v\right|_p$ identifiziert.

\begin{kor}
	Auf dem $\mathbb R^n$ bilden die partiellen Ableitungsoperatoren $\left.\partial_1\right|_p, \dots, \left.\partial_n\right|_p$ eine Basis des Tangentialraums $T_p\mathbb R^n$.

	\begin{proof}
		$\left.\partial_i\right|_p$ ist die Richtungsableitung in Richtung des Standardbasisvektors $\e_i$.
	\end{proof}
\end{kor}

\begin{kor} \label{kor:entw_tangt_Rn}
	Die Entwicklung eines Tangentialvektors $w \in T_p\mathbb R^n$ in dieser Basis ist
	\[w = \sum_{i=1}^n w(\pr^i) \left.\partial_i\right|_p,\]
	wobei $\pr^i\colon \mathbb R^n \to R, \pr^j(a) = a^j$ die Projektion auf die $i$-te Komponente ist.

	\begin{proof}
		Das ergibt sich direkt aus dem Beweis von Satz \ref{satz:tangt_vr}, da die Projektionen $\pr^i$ die zu $\e_i$ duale Basis von $(\mathbb R^n)^*$ bilden.
	\end{proof}
\end{kor}

\begin{bsp}
	Ist $U\subset\mathbb R^n$ offen und $p\in U$, so bilden die partiellen Ableitungsableitungen $\left.\partial_1\right|_p, \dots, \left.\partial_n\right|_p$ eine Basis von $T_pU$ (denn es ist ja $T_pU = T_p\mathbb R^n$).
\end{bsp}

\begin{satz}[Differential von linearen Abbildungen]
	Sei $L\colon V \to W$ eine lineare Abbildung. Bzgl. der Identifikationen $T_pV \equiv V, T_{L(p)}W \equiv W$ ist $\left.\DD L\right|_p = L$.

	\begin{center}
		\begin{tikzpicture}
		\matrix(m)[matrix of math nodes,row sep=4em,column sep=4em,minimum width=2em]{T_pV & T_{L(p)}W \\
			V&W \\};
		\path[-stealth]
		(m-1-1)	edge node [above]{$\left.\DD L\right|_p$} (m-1-2)
		(m-2-1) edge node [left]{$v \mapsto \left.\partial_v\right|_p$} (m-1-1)
		(m-2-2) edge node [right]{$w \mapsto \left.\partial_w\right|_{L(p)}$} (m-1-2)
		(m-2-1) edge node [above]{$L$} (m-2-2)
		;
		\end{tikzpicture}
	\end{center}

	\begin{proof}
		Für $v\in V$ ist
		\begin{align*}
			\left(\left.\DD L\right|_p \left( \left.\partial_v\right|_p \right)\right) (f) &= \left.\partial_v\right|_p (f \circ L)\\
			&= \left.\frac{\D f(L(p + h v))}{\D h}\right|_{h=0}\\
			&= \left.\frac{\D f(L(p) + h L(v))}{\D h}\right|_{h=0}\\
			&= \left.\partial_{L(v)}\right|_{L(p)} (f)
		\end{align*}
		für beliebige $f$, also $\left.\DD L\right|_p \left( \left.\partial_v\right|_p \right) = \left.\partial_{L(v)}\right|_{L(p)}$.
		Das mussten wir aber zeigen.
	\end{proof}
\end{satz}

\begin{bsp}
	Sei $I \subset \mathbb R$ eine offene Menge, $s \in I$ und $\left.\partial_1\right|_s$ der kanonische Ableitungsoperator, der eine Basis von $T_s I$ bildet.

	Ist $M$ eine differenzierbare Mannigfaltigkeit und $\gamma\colon I \to M$ eine glatte Kurve, so können wir das Differential $\left.\DD \gamma\right|_s \colon T_sI \to T_{\gamma(s)} M$ betrachten. Dieses bildet den Basisvektor $\left.\partial_1\right|_s$ ab auf die Derivation $\left.\DD \gamma\right|_s(\left.\partial_1\right|_s)$, die wirkt als $\left(\left.\DD \gamma\right|_s(\left.\partial_1\right|_s)\right)(f) = \left.\partial_1\right|_s (f\circ\gamma) = \left.\frac{\D(f\circ\gamma)(s + h)}{\D h}\right|_{h=0} = (f\circ\gamma)'(s)$. Das ist aber genau die Wirkung des in Beispiel \ref{bsp:kurve_geschw} definierten \glqq Geschwindigkeitsvektors\grqq\ $\gamma'(s) \in T_{\gamma(s)} M$, also gilt $\left.\DD \gamma\right|_s(\left.\partial_1\right|_s) = \gamma'(s)$.

	Der \glqq Geschwindigkeitsvektor\grqq\ $\gamma'(s)$ ist also tatsächlich in einem formalen Sinne die Ableitung der Kurve $\gamma$ an der Stelle $s$, nämlich das Differential von $\gamma$ an dieser Stelle angewendet auf den kanonischen Basisvektor von $T_sI$.
\end{bsp}

\begin{lemma}[Differential von glatten Funktionen als Anwendung der Derivation]
	Für $f\in\sC(M)$ und $v \in T_pM$ ist $\left.\DD f\right|_p(v) = v(f)$ unter der kanonischen Identifikation $T_{f(p)}\mathbb R \equiv \mathbb R$.

	\begin{proof}
		Nach Korollar \ref{kor:entw_tangt_Rn} ist $\left.\DD f\right|_p(v) = \Big(\DD f\big|_p(v)\Big)(\id) \cdot \left.\partial_1\right|_{f(p)} \in T_{f(p)}\mathbb R$. Da nach Definition $\Big(\DD f\big|_p(v)\Big)(\id) = v(\id \circ f) = v(f)$ gilt und $\left.\partial_1\right|_{f(p)}$ unter $T_{f(p)}\mathbb R \equiv \mathbb R$ mit $1$ identifiziert wird, folgt die Behauptung.
	\end{proof}
\end{lemma}

%%%%%%%%%%%%%%%%%%%%%%%%%%%%%%%%%%%%%%%%%%%%%%%%%%%%%%%%%%%
\section{Koordinatenvektoren}

Haben wir auf einer Mannigfaltigkeit $M$ eine Karte $(x,U)$ gegeben, dann ermöglicht sie uns, $U$ mit einer offenen Teilmenge des $\mathbb R^n$ zu identifizieren. Das gilt auch für die Tangentialräume:
\begin{defn}
	Sei $M$ eine differenzierbare Mannigfaltigkeit, $p\in M$ und $(x,U)$ eine Karte um $p$. Wir schreiben $V = x(U) \subset \mathbb R^n$. Als Abbildung $x\colon U \to V$ ist $x$ ein Diffeomorphismus (Übung!). Nach Lemma \ref{lemma:differential_diffeo} ist dann also
	\[\left.\DD x\right|_p \colon T_pM \to T_{x(p)} \mathbb R^n\]
	ein Isomorphismus, wobei wir die Identifikationen $T_pM = T_pU$, $T_{x(p)}\mathbb R^n = T_{x(p)} x(U)$ benutzt haben. Diesen benutzen wir, um aus der kanonischen Basis $\{\left.\partial_i\right|_{x(p)}\}$ von $T_{x(p)}\mathbb R^n$ eine Basis von $T_pM$ zu erhalten: Für $i = 1,\dots, n$ definieren wir
	\[\left.\frac{\partial}{\partial x^i}\right|_p := \left(\left.\DD x\right|_p\right)^{-1} \left(\left.\partial_i\right|_{x(p)}\right) = \left.\DD x^{-1}\right|_{x(p)} \left(\left.\partial_i\right|_{x(p)}\right)\]
	Die $\left.\frac{\partial}{\partial x^i}\right|_p$ heißen die von $x$ \emph{induzierten Koordinatenvektoren} in $p$, zusammen bilden sie die von $x$ induzierte \emph{Koordinatenbasis} von $T_pM$.
\end{defn}
Die Koordinatenvektoren wirken durch
\[\left.\frac{\partial}{\partial x^i}\right|_p(f) = \left.\partial_i\right|_{x(p)} (f \circ x^{-1}) = \left(\partial_i (f\circ x^{-1})\right) (x(p)).\]
$\left.\frac{\partial}{\partial x^i}\right|_p$ stellt also einen Funktionskeim in der Karte $x$ dar und bildet die $i$-te partielle Ableitung dieser Kartendarstellung.

\begin{bem}
	Wenn wir in Koordinaten rechnen wollen, schreiben wir Karten meist als $x = (x^1, \dots, x^n)$ o.\,ä. statt als $\varphi$ o.\,ä., da das zur historisch etablierten Notation von Koordinatenvektoren passt. Die Funktionen $x^i$ nennt man dann auch \emph{Koordinatenfunktionen}, die $x^i(p)$ sind die \emph{Koordinaten} von $p$.

	Aufpassen muss man dann allerdings manchmal, wenn man Punkte im $\mathbb R^n$ auch als $x$ schreibt (deshalb haben wir das oben auch eher vermieden).
\end{bem}

\begin{bsp}
	Für eine Karte $(x,U)$ um $p$ sind die Komponenten $x^i, i = 1,\dots, n$ reellwertige glatte Funktionen auf $U$, können also als Funktionskeime in $p$ aufgefasst werden. Wendet man darauf die Koordinatenvektoren an, so erhält man
	\begin{align*}
		\left.\frac{\partial}{\partial x^i}\right|_p (x^j) &= \left(\partial_i \left(x^j\circ x^{-1}\right)\right) (x(p))\\
		&= \underbrace{\left(\partial_i \pr^j\right)}_{= \delta^j_i} (x(p)) = \delta^j_i,
	\end{align*}
	wobei $\pr^j\colon \mathbb R^n \to R, \pr^j(a) = a^j$ die Projektion auf die $j$-te Komponente ist.
\end{bsp}

\begin{lemma}[Koordinatendarstellung von Tangentialvektoren]
	Sei $M$ eine differenzierbare Mannigfaltigkeit, $p \in M$, $v \in T_pM$ und $x$ eine Karte um $p$. Die Darstellung von $v$ in der von $x$ induzierten Koordinatenbasis ist
	\[v = \sum_{i=1}^n v^i \left.\frac{\partial}{\partial x^i}\right|_p\]
	mit
	\[v^i = v(x^i).\]

	\begin{proof}
		Nach Korollar \ref{kor:entw_tangt_Rn} ist $\left.\DD x\right|_p(v) = \sum_{i=1}^n \Big(\DD x\big|_p(v)\Big) (\pr^i) \left.\partial_i\right|_{x(p)}$. Da nach Definition $\Big(\DD x\big|_p(v)\Big) (\pr^i) = v(\pr^i \circ x) = v(x^i)$ ist, haben wir also $\left.\DD x\right|_p(v) = \sum_{i=1}^n v(x^i) \left.\partial_i\right|_{x(p)}$.
		Anwenden von $\left(\left.\DD x\right|_p\right)^{-1}$ auf beiden Seiten liefert die Behauptung.
	\end{proof}
\end{lemma}

\begin{lemma}[Differential in Kartendarstellung] \label{lemma:differential_koord}
	Sei $f\colon M \to N$ eine glatte Abbildung zwischen differenzierbaren Mannigfaltigkeiten. Sei $p\in M$, $x$ eine Karte von $M$ um $p$ und $y$ eine Karte von $N$ um $f(p)$. Dann ist die Darstellungsmatrix des Differentials $\left.\DD f\right|_p$ bzgl. der Koordinatenbasen $\left.\frac{\partial}{\partial x^i}\right|_p$ von $T_pM$ und $\left.\frac{\partial}{\partial y^i}\right|_{f(p)}$ von $T_{f(p)}N$ die Jacobi-Matrix der Kartendarstellung $y \circ f \circ x^{-1}$ von $f$ an der Stelle $x(p)$, also die Matrix
	\[\mathrm{J}(y\circ f\circ x^{-1})|_{x(p)} = \left[\left(\partial_j(y\circ f\circ x^{-1})^i\right) (x(p)) \right]_{ij}.\]
	\bewUeb
\end{lemma}

\begin{defn}
	Sei $M$ eine differenzierbare Mannigfaltigkeit und $(x,U)$ eine Karte. Für $f \in \sC(U)$ definieren wir die Funktionen $\frac{\partial f}{\partial x^i} \in \sC(U)$ durch
	\[\frac{\partial f}{\partial x^i}(p) := \left.\frac{\partial}{\partial x^i}\right|_p(f).\]
	Manchmal nennen wir diese Funktionen die \emph{partiellen Ableitungen} von $f$ bzgl. der Karte $x$, wobei das natürlich \emph{abuse of language} ist.
\end{defn}

Mit dieser Notation ist die Darstellungsmatrix des Differentials aus Lemma \ref{lemma:differential_koord} also
\[\left[\frac{\partial(y^i \circ f)}{\partial x^j}(p)\right]_{ij}.\]

\begin{lemma}[Transformationsformel für Tangentialvektoren]
	Sei $M$ eine differenzierbare Mannigfaltigkeit, $p\in M$ und seien $x, y$ zwei Karten um $p$. Für die Koordinatenvektoren gilt
	\[\left.\frac{\partial}{\partial y^i}\right|_p = \sum_{j=1}^n \frac{\partial x^j}{\partial y^i}(p) \left.\frac{\partial}{\partial x^j}\right|_p.\]
	Die Basiswechselmatrix ist also die Jacobi-Matrix des inversen Kartenwechsels $x \circ y^{-1}$ (an der Stelle $y(p)$).

	Für einen Tangentialvektor $v = \sum_{i=1}^n v^i \left.\frac{\partial}{\partial x^i}\right|_p = \sum_{i=1}^n \tilde v^i \left.\frac{\partial}{\partial y^i}\right|_p$ gilt dementsprechend
	\[\tilde v^j = \sum_{i=1}^n \frac{\partial y^j}{\partial x^i}(p) v^i.\]

	\bewUeb
\end{lemma}

\todo{Summenkonvention einführen.}

%%%%%%%%%%%%%%%%%%%%%%%%%%%%%%%%%%%%%%%%%%%%%%%%%%%%%%%%%%%
\section{Ableitungen von Kurven}
\begin{bsp}
	Für eine offene Menge $I\subset\mathbb R$ schreibt man die kanonische Karte $\id_I\colon I \to I$ oft als $t$ und den davon induzierten Koordinatenvektor dann als $\left.\frac{\D}{\D t}\right|_s \in T_sI$. Das ist natürlich einfach der normale Ableitungsoperator an der Stelle $s$, den wir oben schon als $\left.\partial_1\right|_s$ geschrieben haben. Für eine glatte Kurve $\gamma\colon I \to M$ ist also
	\[\gamma'(s) = \left.\DD\gamma\right|_s \left(\left.\frac{\D}{\D t}\right|_s\right).\]

	Statt $\gamma'(s)$ schreiben wir für die Ableitung einer Kurve an der Stelle $s$ in Zukunft auch
	\[\gamma'(s) = \left.\frac{\D}{\D t}\gamma(t)\right|_{t = s}.\]

	Man kann sich überlegen, dass
	\[\gamma'(s) = \left.\frac{\D}{\D t}\gamma(s + t)\right|_{t = 0}\]
	gilt (Übung!).
\end{bsp}
\todo{Ausarbeiten! Zeigen: Jeder Tangentialvektor ist Ableitung einer Kurve. Koordinatenvektoren sind Ableitungen von Koordinatenlinien.}

\section{Untermannigfaltigkeiten}
\kommP{Weglassen? Brauchen wir nicht später, und wir haben eh zu wenig Zeit, und das Kapitel ist eh zu voll ...}


%%%%%%%%%%%%%%%%%%%%%%%%%%%%%%%%%%%%%%%%%%%%%%%%%%%%%%%%%%%
\chapter{Vektorbündel}
	\section{Faserbündel}
		\begin{defn}[Faserbündel]
			Ein \emph{Faserbündel} $(P,M,\pi ,F)$ ist ein Tupel aus drei Mannigfaltigkeiten (dem \emph{Totalraum} $P$, der \emph{Basis} $M$ und der \emph{typischen Faser} $F$) zusammen mit einer surjektiven \emph{Projektion} $\pi\colon P\rightarrow M$, die die folgenden Eigenschaften erfüllt:

				Für alle Punkte $x\in M$ existieren eine offene Umgebung $x\in U\subset M$ und ein Diffeomorphismus $\phi\colon \pi^{-1}(U)\rightarrow U\times F$, sodass das folgende Diagramm kommutiert, also $\pi\vert_{\pi^{-1}(U)}=\pr_1\circ\phi$ (wobei $\pr_1$ die Projektion auf den ersten Faktor bezeichnet):
			\begin{center}
			\begin{tikzpicture}
				\matrix(m)[matrix of math nodes,row sep=3em,column sep=4em,minimum width=2em]{\pi^{-1}(U) & U\times F \\
     U& \\};
  				\path[-stealth]
  					(m-1-1)	edge node [above]{$\phi$} (m-1-2)
  							edge node [left]{$\pi$} (m-2-1)
  					(m-1-2) edge node [below]{$\quad\pr_1$} (m-2-1)
  					;
			\end{tikzpicture}
			\end{center}
			Eine solche Abbildung $\phi$ heißt \emph{lokale Trivialisierung}. Aufgrund der Existenz lokaler Trivialisierungen ist das Urbild jedes Punktes $x\in M$ unter der Projektion (\emph{Faser} von $x$ genannt) $P_x:=\pi^{-1}(\lbrace x\rbrace )$ diffeomorph zur typischen Faser: $P_x\cong F \; \forall x\in M$.

			Eine Familie lokaler Trivialisierungen, die $P$ überdecken, heißt \emph{Bündelatlas}.
		\end{defn}
		\begin{nota}
			Man sagt auch \glqq Faserbündel über $M$\grqq und schreibt dafür einfach $(P,\pi)$.
		\end{nota}
		Sei von nun an $(P,M,\pi ,F)$ ein Faserbündel.
		\begin{bsp}\hfill
			\begin{enumerate}
				\item Das triviale Bündel mit typischer Faser $F$ und Basis $M$ ist das kartesische Produkt $M\times F$ (genauer gesagt das Bündel $(M\times F,M,\pr_1,F))$.
				\item Für eine offene Menge $U\subset M$ und $P\vert_U:=\pi^{-1}(U)$ ist $(P\vert_U,U,\pi\vert_{P_U},F)$ ein Faserbündel.
			\end{enumerate}
		\end{bsp}
		\begin{defn}[Schnitt]
			Eine Abbildung $\sigma\colon M\rightarrow P$ heißt (globaler) \emph{Schnitt}, wenn $\pi\circ\sigma=\id_M$, also $\sigma(x)\in P_x\, \forall x\in M$. Analog heißt für eine offene Menge $U\subset M$ eine Abbildung $\sigma\colon U\rightarrow P\vert_U$ \emph{lokaler Schnitt}, wenn $\pi\circ\sigma=\id_U$.

			Die Menge aller globalen Schnitte wird mit $\Gamma(P)$ bezeichnet, die der lokalen Schnitte mit $\Gamma(P\vert_U)$.
		\end{defn}
	\section{Vektorbündel}
		\begin{defn}[Vektorbündel]
				Ein Faserbündel $(E,M,\pi,\R^n)$ heißt \emph{Vektorbündel} vom \emph{Rang} $n$, wenn $E_x$ für alle Punkte $x\in M$ ein $n$-dimensionaler Vektorraum ist. Weiter soll es einen Bündelatlas aus lokalen Trivialisierungen ${\phi\colon \pi^{-1}(U)\rightarrow U\times \R^n}$ geben, sodass die folgende Abbildung für jeden Punkt $x\in U$ ein Vektorraumisomorphismus ist:
				\begin{equation*}
					E_x\rightarrow \R^n,\; p\mapsto \pr_2(\phi(p))
				\end{equation*}
			Daraus folgt, dass die Abbildung ${\R^n\rightarrow E_x,\; v\mapsto \phi^{-1}(x,v)}$ ebenfalls ein Vektorraumisomorphismus ist.
		\end{defn}
		Analog zu Kartenwechseln lassen sich auch \emph{Trivialisierungswechsel} definieren:
		\begin{defn}
			Sei $(E,M,\pi,\R^n)$ ein Vektorbündel, $U,V\subset M$ mit $U\cap V\neq\emptyset$ und Trivialisierungen ${\phi_U\colon \pi^{-1}(U)\rightarrow U\times \R^n}, {\phi_V\colon \pi^{-1}(U)\rightarrow U\times \R^n}$.

			Die folgende Abbildung heißt \emph{Trivialisierungswechsel}:
			\begin{equation*}
				\phi_V\circ\phi_U^{-1}\colon (U\cap V)\times \R^n\rightarrow (U\cap V)\times \R^n, (q,v)\mapsto (q,\tau(q)v)
			\end{equation*}
			wobei die Einschränkung unterdrückt wurde. Die Abbildung $\tau\colon U\cap V\rightarrow \GL(n,\R)$ ist glatt und wird als \emph{Übergangsfunktion} bezeichnet.
		\end{defn}
		\begin{defn}[Vektorbündelhomomorphismen]\hfill\\
			Seien $(E_a,M_a,\pi_a,\R^{n_a}),(E_b,M_b,\pi_b,\R^{n_b})$ Vektorbündel.
			
			Ein Tupel $(F,f)$ mit $F\colon E_a\rightarrow E_b$ und $f\colon M_a\rightarrow M_b$ heißt \emph{Vektorbündelhomomorphismus}, wenn das folgende Diagramm kommutiert:
			\begin{center}
				\begin{tikzpicture}
					\matrix(m)[matrix of math nodes,row sep=3em,column sep=4em,minimum width=2em]{E_a & E_b \\
     M_a&M_b \\};
  				\path[-stealth]
  					(m-1-1)	edge node [above]{$F$} (m-1-2)
  							edge node [left]{$\pi_a$} (m-2-1)
  					(m-1-2) edge node [right]{$\pi_b$} (m-2-2)
  					(m-2-1) edge node [above]{$f$} (m-2-2)
  					;
				\end{tikzpicture}
			\end{center}
			also $\pi_b\circ F=f\circ\pi_a$.
		\end{defn}
		\begin{bsp}\hfill
			\begin{enumerate}
				\item Das triviale Bündel $M\times \R^n$ ist ein Vektorbündel vom Rang $n$.
				\item Das Tangentialbündel $TM:=\bigcup_{x\in M}T_xM$ ist ein Vektorbündel vom Rang $n:=\dim M$. Beachte jedoch, dass im Allgemeinen $TM\ncong M\times \R^n$ (es existiert also nicht unbedingt ein Vektorbündelisomorphismus zwischen dem Tangential- und dem trivialen Bündel). Im Fall $TM\cong M\times \R^n$ heißt die Mannigfaltigkeit parallelisierbar.
			\end{enumerate}
		\end{bsp}
	\section{Konstruktionen von Vektorbündeln}
		Seien $(E_a,M,\pi_a,\R^{n_a}),(E_b,M,\pi_b,\R^{n_b})$ Vektorbündel.\\
		\begin{defn}
			Das zu $(E,M,\pi,\R^{n})$ \emph{duale Bündel} $(E^{\prime},M,\pi^{\prime},(\R^n)^{\prime})$ ist gegeben durch
			\begin{align*}
				E^{\prime}&:=\bigcup_{x\in M}(E_{x})^{\prime}\\
				\pi^{\prime}&\colon E^{\prime}\rightarrow M\\
				(\pi^{\prime})^{-1}(x)&=E^{\prime}_x=(E_{x})^{\prime}\quad \forall x\in M
			\end{align*}
			und für eine Trivialisierung $(\phi,U)$ von $E$, ist die folgende Abbildung eine Trivialisierung des dualen Bündels:
			\begin{align*}
				\phi^{\prime}\colon(\pi^{\prime})^{-1}(U)&\rightarrow U\times (\R^n)^{\prime}\\
				\lambda&\mapsto\left(x,\lambda\circ(\pr_2\circ\phi)^{-1}\right)
			\end{align*}
			wobei $\lambda\in E_x^{\prime}$. Trivialisierungswechsel sind (unter Identifikation von $\R^n$ mit Spalten- und $(\R^n)^{\prime}$ mit Zeilenvektoren) durch $\tau^{\prime}=\tau^{-1}$ gegeben, sodass $(x,\lambda)\mapsto(x,\lambda\tau^{\prime}(x))$.
		\end{defn}
		\begin{bsp}
			Mit obiger Konstruktion können wir aus dem Tangentialbündel $TM$ das \emph{Kotangentialbündel} $T^{\prime}M$ gewinnen. Wie wir später sehen werden, ergeben sich in Kombination mit den folgenden Konstruktionen viele neue Strukturen.
		\end{bsp}
		\begin{defn}[Whitney-Summe]
			Die \emph{Whitney-Summe} $(E_a\oplus E_b ,M,\pi,\R^{n_a+n_b})$ ist ein Vektorbündel gegeben durch
			\begin{align*}
				E_a\oplus E_b&:=\bigcup_{x\in M}E_{a,x}\oplus E_{b,x}\\
				\pi&\colon E_a\oplus E_b\rightarrow M\\
				(\pi)^{-1}(x)&=(E_a\oplus E_b)_x=E_{a,x}\oplus E_{b,x}\quad \forall x\in M
			\end{align*}
			und für Trivialisierungen $(\phi_a,U),(\phi_b,U)$ ist die folgende Abbildung eine Trivialisierung der Whitney-Summe:
			\begin{align*}
				\phi\colon\pi^{-1}(U)&\rightarrow U\times \R^{n_a+n_b}\\
				v_a\oplus v_b&\mapsto (\pi_a(v_a), \pr_2(\phi_a(v_a))\oplus\pr_2(\phi_b(v_b)))
			\end{align*}
			Trivialisierungswechsel sind (wieder unter Verwendung von Spaltenvektoren) durch die folgende Blockdiagonalmatrix gegeben: 
			\begin{equation}
				\tau=\left(\begin{array}{cc}\tau_a&0\\0&\tau_b\\ \end{array}\right)
			\end{equation}
		\end{defn}
		\begin{defn}[Tensorprodukt]
			Das \emph{Tensorprodukt} $(E_a\otimes E_b ,M,\pi,\R^{n_a\cdot n_b})$ von Vektorbündeln ist ein Vektorbündel gegeben durch
			\begin{align*}
				E_a\otimes E_b&:=\bigcup_{x\in M}E_{a,x}\otimes E_{b,x}\\
				\pi&\colon E_a\otimes E_b\rightarrow M\\
				(\pi)^{-1}(x)&=(E_a\otimes E_b)_x=E_{a,x}\otimes E_{b,x}\quad \forall x\in M
			\end{align*}
			und für Trivialisierungen $(\phi_a,U),(\phi_b,U)$ ist die folgende Abbildung eine Trivialisierung des Tensorprodukts:
			\begin{align*}
				\phi_a\otimes\phi_b\colon\pi^{-1}(U)&\rightarrow U\times \R^{n_a\cdot n_b}\\
				v_a\otimes v_b&\mapsto (\pi_a(v_a), \pr_2(\phi_a(v_a))\otimes\pr_2(\phi_b(v_b)))
			\end{align*}
			Die Übergangsfunktionen des Tensorproduktes sind durch das Tensorprodukt der Übergangsfunktionen gegeben: $\tau=\tau_a\otimes\tau_b$.
		\end{defn}
		\todo{Tensorprodukt von VR}
		
%%%%%%%%%%%%%%%%%%%%%%%%%%%%%%%%%%%%%%%%%%%%%%%%%%%%%%%%%%%
\chapter{Vektorfelder}

%%%%%%%%%%%%%%%%%%%%%%%%%%%%%%%%%%%%%%%%%%%%%%%%%%%%%%%%%%%
\section{Grundlegendes}
Definition: Schnitt im Tangentialbündel. Zeigen, dass äquivalent zu (globalen) Derivationen von $\sC(M)$. (Dazu vorher als Lemma: Punktweise Derivationen von $\sC(M)$ lassen sich zu Derivationen von Keimen \glqq lokalisieren\grqq\ via bump functions.)

%%%%%%%%%%%%%%%%%%%%%%%%%%%%%%%%%%%%%%%%%%%%%%%%%%%%%%%%%%%
\section{Pushforward und Kommutator}
Definitionen

%%%%%%%%%%%%%%%%%%%%%%%%%%%%%%%%%%%%%%%%%%%%%%%%%%%%%%%%%%%
\section{Integralkurven und die Lie-Ableitung}
Definition von Integralkurven, Existenz per Picard-Lindelöf. Definition Fluss, Eigenschaften. Defn. Lie-Ableitung (für Funktionen und Vektorfelder). Zeigen, dass Ableitung / Kommutator.


%%%%%%%%%%%%%%%%%%%%%%%%%%%%%%%%%%%%%%%%%%%%%%%%%%%%%%%%%%%
\chapter{Lie-Gruppen}
	\section{Grundlagen}
		\begin{defn}
			Eine \emph{Lie-Gruppe} ist eine differenzierbare Mannigfaltigkeit $G$ mit einer Gruppenstruktur, sodass die Verknüpfung $\cdot\colon G\times G\rightarrow G$ und die Inversion $i\colon G\rightarrow G, g\mapsto g^{-1}$ glatte Abbildungen sind.
		\end{defn}

%%%%%%%%%%%%%%%%%%%%%%%%%%%%%%%%%%%%%%%%%%%%%%%%%%%%%%%%%%%
\chapter{Tensorfelder}


%%%%%%%%%%%%%%%%%%%%%%%%%%%%%%%%%%%%%%%%%%%%%%%%%%%%%%%%%%%
\chapter{Differentialformen}


%%%%%%%%%%%%%%%%%%%%%%%%%%%%%%%%%%%%%%%%%%%%%%%%%%%%%%%%%%%
\chapter{Kovariante Ableitungen}


%%%%%%%%%%%%%%%%%%%%%%%%%%%%%%%%%%%%%%%%%%%%%%%%%%%%%%%%%%%
\chapter{Riemann'sche Geometrie}


%%%%%%%%%%%%%%%%%%%%%%%%%%%%%%%%%%%%%%%%%%%%%%%%%%%%%%%%%%%
\chapter{Integration auf Mannigfaltigkeiten}


\end{document}