% !TeX spellcheck = de_DE
\documentclass[a4paper]{scrartcl}
\usepackage[utf8]{inputenc}
\usepackage[T1]{fontenc}
\usepackage{lmodern}
\usepackage[sc]{mathpazo}
\linespread{1.05}
\usepackage{exscale} % To scale mathematical symbols correctly while using T1
\usepackage{amsmath,amsthm,amssymb,dsfont}
\usepackage[ngerman]{babel}
\usepackage{enumitem}
\usepackage{microtype} % Microtypography!
\usepackage{hyperref}

\DeclareMathOperator{\supp}{supp}

\newcommand{\R}{\mathbb{R}}
\newcommand{\sC}{\mathcal{C}^{\infty}}

\theoremstyle{definition}
\newtheorem{defn}{Definition}
\newtheorem{lemma}[defn]{Lemma}
\newtheorem{satz}[defn]{Satz}
\newtheorem{bem}[defn]{Bemerkung}

\title{Details zur Existenz von Zerlegungen der Eins}
\subtitle{Einführung in die Differentialgeometrie}
\author{Benjamin Haake, Philip Schwartz}
\date{}

\begin{document}

\maketitle

Hier geben wir ein paar mehr Details zur Existenz von Zerlegungen der Eins. Dabei orientieren wir uns stark am entsprechenden Abschnitt im Buch von Lee.

\begin{defn}
	Sei $M$ ein topologischer Raum und $\{X_i\}_{i\in I}$ eine Überdeckung von $M$. Eine \emph{Verfeinerung} dieser Überdeckung ist eine Überdeckung $\{Y_j\}_{j\in J}$, sodass jedes $Y_j$ in einem der $X_i$ enthalten ist.
\end{defn}

\begin{lemma} \label{lemma:lok_endl_abschluss}
	Sei $\{X_i\}_{i\in I}$ eine Familie von Teilmengen eines topologischen Raums $M$. Ist $\{X_i\}_{i\in I}$ lokal endlich, so ist $\overline{\bigcup_{i\in I} X_i} = \bigcup_{i\in I} \overline{X_i}$. Außerdem ist auch $\{\overline{X_i}\}_{i\in I}$ lokal endlich.

	(ohne Beweis)
\end{lemma}

Zur Erinnerung hier nochmal die Definition von Zerlegungen der Eins:
\begin{defn}
	Sei $M$ eine differenzierbare Mannigfaltigkeit und $\{U_i\}_{i\in I}$ eine offene Überdeckung von $M$. Eine $\{U_i\}_{i\in I}$ \emph{untergeordnete Zerlegung der Eins} ist eine Familie $\{\psi_i\}_{i\in I}$ von glatten Funktionen $\psi_i \in \sC(M)$ mit folgenden Eigenschaften:
	\begin{enumerate}[label=(\roman*)]
		\item Die $\psi_i$ nehmen Werte in $[0,1]$ an.
		\item Für jedes $i\in I$ ist $\supp(\psi_i) \subset U_i$.
		\item $\{\supp(\psi_i)\}_{i\in I}$ ist lokal endlich.
		\item \label{defn:zerl_eins_summe} Für alle $p\in M$ ist $\sum_{i\in I} \psi_i(p) = 1$.
	\end{enumerate}
\end{defn}

Wie im Haupttext gesagt, ist die Idee des Beweises der Existenz von Zerlegungen der Eins, die $\psi_i$ aus bump functions auf Kugeln zu konstruieren, die Bilder von Kartenumgebungen in den gegebenen offenen Mengen sind. Um das auszuführen, brauchen wir noch einen kleinen technischen Hilfssatz:
\begin{lemma}
	Sei $M$ eine differenzierbare Mannigfaltigkeit und $\{U_i\}_{i\in I}$ eine offene Überdeckung von $M$. Es existieren eine Familie $\{(\varphi_j,V_j)\}_{j\in J}$ von Karten von $M$, die $M$ überdecken, sowie eine weitere offene Überdeckung $\{B_j\}_{j\in J}$ von $M$ mit den folgenden Eigenschaften:
	\begin{enumerate}[label=(\roman*)]
		\item $\{V_j\}_{j\in J}$ ist eine Verfeinerung von $\{U_i\}_{i\in I}$.
		\item Für alle $j\in J$ ist $\bar B_j \subset V_j$.
		\item $\{B_j\}_{j\in J}$ ist lokal endlich.
		\item Zu jedem $j\in J$ existiert ein $r_j > 0$ mit $\varphi_j(B_j) = B_{r_j}(0)$ (d.\,h. die Karte bildet $B_j$ auf eine Kugel ab).
	\end{enumerate}

	(ohne Beweis, da zu technisch/topologisch)
\end{lemma}
\begin{bem}
	Dieses Lemma zeigt insbesondere, dass Mannigfaltigkeiten \emph{parakompakt} sind: Jede offene Überdeckung hat eine lokal endliche Verfeinerung.
\end{bem}

\begin{satz}
	Zu jeder offenen Überdeckung $\{U_i\}_{i\in I}$ einer differenzierbaren Mannigfaltigkeit $M$ gibt es eine untergeordnete Zerlegung der Eins.

	\begin{proof}
		Wende das vorige Lemma an, um $\{(\varphi_j,V_j)\}_{j\in J}$, $\{B_j\}_{j\in J}$, $\{r_j\}_{j\in J}$ wie dort zu erhalten. Für jedes $j\in J$ wählen wir eine bump function $H_j \in \sC(\R^n)$, die auf $B_{r_j}(0)$ positiv ist und außerhalb von $B_{r_j}(0)$ verschwindet (diese existiert nach Lemma \ref{lemma:bump_fn_kugel} für $r' = r_j$).

		Für $j\in J$ definieren wir $f_j \in \sC(M)$ durch
		\[f_j := \begin{cases}
		H_j \circ \varphi_j & \text{auf} \; V_j\\
		0 & \text{auf} \; M \setminus \overline{B_j}.
		\end{cases}\]
		Auf dem Durchschnitt $V_j \setminus \overline{B_j}$ der beiden Definitionsbereiche ergeben beide Ausdrücke 0, $f_j$ ist also wohldefiniert. Auf $B_j$ ist $f_j$ positiv, außerhalb von $B_j$ ist es 0; also ist $\supp(f_j) = \bar B_j$.

		Wir definieren jetzt $F \in \sC(M)$ durch $F(p) := \sum_{j\in J} f_j(p)$ (wegen der lokalen Endlichkeit von $\{B_j\}_{j\in J}$ existiert die Summe). Da die $f_j$ nichtnegativ und auf $B_j$  jeweils positiv sind, ist $F > 0$ (da jedes $p\in M$ in einem $B_j$ liegt). Wenn wir also $g_j := f_j / F \in \sC(M)$ setzen, dann ist jedes $g_j$ eine nichtnegative Funktion, die auf $B_j$ positiv ist und für die $\supp(g_j) = \bar B_j$ gilt, und es ist $\sum_{j \in J} g_j = F/F = 1$.

		Jetzt müssen wir die konstruierten Funktionen $g_j$ noch zu den gesuchten Funktionen $\{\psi_i\}_{i\in I}$ zusammenfassen. Da $\{V_j\}_{j\in J}$ eine Verfeinerung von $\{U_i\}_{i\in I}$ ist, gibt es zu jedem $j \in J$ ein $\lambda(j) \in I$ mit $V_j \subset U_{\lambda(j)}$. Wir setzen
		\[\psi_i := \sum_{j\in J: \lambda(j) = i} g_j \in \sC(M).\]
		Weil die $g_j$ positiv sind und jedes $g_j$ genau auf $B_j$ von 0 verschieden ist, gilt jeweils $\supp(\psi_i) = \overline{\bigcup_{j: \lambda(j) = i} B_j} = \bigcup_{j: \lambda(j) = i} \overline{B_j} \subset U_i$, wobei bei der letzten Gleichheit Lemma \ref{lemma:lok_endl_abschluss} benutzt wurde. Nach diesem Lemma ist außerdem $\{\overline{B_j}\}_{j\in J}$ lokal endlich; da die $\overline{B_j}$ einfach nur zu den Trägern $\supp(\psi_i)$ \glqq zusammengefasst werden\grqq, ist damit auch $\{\supp(\psi_i)\}_{i\in I}$ lokal endlich.

		Nach Konstruktion nehmen die $\psi_i$ Werte in $[0,1]$ an und es gilt $\sum_{i\in I} \psi_i = 1$.
	\end{proof}
\end{satz}

\end{document}